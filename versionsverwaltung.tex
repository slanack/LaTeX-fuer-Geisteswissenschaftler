% !TeX root = lfgw.tex
\chapter{Versionsverwaltung mit Subversion und Git}
% Uwe Ziegenhagen

Beim Schreiben längerer Dokumente über Wochen und Monate hinweg sollte man sich auf jeden Fall um eine zuverlässige Datensicherung kümmern. Die Kosten einer professionellen Datenrettung gehen schnell in den vierstelligen Bereich, wenn eine Festplatte kaputt geht, im Falle der Verlusts oder Diebstahls des Rechners sind die Daten ohne Sicherung schlichtweg fort und müssen neu erstellt werden.

Die übliche Vorgehensweise, das Sichern von Dateien mit unterschiedlichen Zeitstempeln im Namen, ist nicht zu empfehlen. Sie schützt nur gegen Fehler, die man selbst im Dokument einbaut, nicht gegen Verlust oder Defekt. Auch gegen einen Verschlüsselungstrojaner, der alle Dateien auf dem Rechner verschlüsselt und -- falls man Glück hat -- gegen anonyme Zahlung eines signifikanten Geldbetrages wieder entschlüsselt, hilft dies nicht.

Die aus Sicht des Autors beste Lösung ist daher eine Sicherung außerhalb des lokalen Rechners, idealerweise auch auf einem anderen Dateisystem. Der Grund hierfür ist technisch: Es sind Fälle bekannt, in denen Trojaner nicht nur die lokalen Dateien sondern auch alle Dateien auf angeschlossenen Netzlaufwerken verschlüsselten. Eine Sicherung auf einer Netzwerkfreigabe kann daher auch zu wenig sein. 

In der Softwareentwicklung setzt man seit Jahrzehnten auf die Nutzung von Versionsverwaltungssystem wie CVS/RSC, Subversion und Git, die es neben dem Sicherungsaspekt auch erlauben, zu früheren Versionen eines Dokuments zurückzukehren. Konsequent genutzt kann man mit einer Versionsverwaltung beispielsweise zur letzten \TeX-Datei zurückgehen, die sich noch problemlos übersetzen ließ und diese mit der aktuellen, eventuell fehlerhaften, Version vergleichen. Und noch ein weiterer Aspekt soll genannt werden: die Arbeit mit mehreren Autoren und/oder auf mehreren Rechnern. Der Autor dieses Kapitels speichert wichtige Dokumente in einem Subversion-Repository. Soll mit einem anderem Rechner an den Dateien gearbeitet werden, so wird dort einfach ein neuer Checkout gemacht. Es darf nur am Ende der Arbeit nicht vergessen werden, die Dateien auch wieder mit dem Repository abzugleichen, dieser wichtige Schritt geht einem aber nach kurzer Zeit \enquote{in Fleisch und Blut} über.

Die Zusammenarbeit in einem Team vereinfacht sich drastisch, da der mühsame und fehlerträchtige Austausch von Dateien über FTP oder Email entfällt, ältere Versionen einer Datei können problemlos wiederhergestellt werden und die Zusammenführung verschiedener Versionen vereinfacht sich. Ein nicht zu unterschätzender Vorteil ist zudem die Möglichkeit, Backups quasi im \enquote{Vorbeigehen} zu erstellen.

In diesem Kapitel soll am Beispiel von Subversion und Git gezeigt werden, wie auch beim bei der Arbeit mit \TeX/\LaTeX\ Versionsverwaltungssysteme sinnvoll eingesetzt werden können.

Da die Installation und Konfiguration beider Systeme einen gewissen technischen Sachverstand voraussetzt, soll an dieser Stelle nur der fachliche Umgang gezeigt werden, nicht die Installation. Wer einen eigenen Subversion- oder Git-Server betreiben möchte, findet im Internet eine Vielzahl von Tutorials. Für die folgenden Beispiele wird jeweils ein Subversion/Git Server genutzt, der auf einem privaten NAS (\enquote{Network-Attached Storage}) der Firma Synology läuft und Backups in der Amazon-Cloud ablegt. Fr beide Versionsverwaltungen gibt es auch eine Vielzahl kommerzieller Anbieter, die die notwendige Infrastruktur bereitstellen. Github beispielsweise berechnet 7~US-\$ pro Monat für eine unbegrenzte Zahl privater (nicht öffentlich einsehbarer) Repositories, LCube-Hosting (\url{https://www.lcube-webhosting.de}) berechnet 2,99~Euro pro Monat für ein Gigabyte an quasi unbegrenzten Subversion-Repositories.

\section{Subversion}



\section{Git}


\endinput


