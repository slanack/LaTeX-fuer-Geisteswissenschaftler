% !TeX root = lfgw.tex
\chapter{Versionsverwaltung mit Subversion und Git}
\autor{Uwe Ziegenhagen}

Beim Arbeiten an längeren oder sehr wichtigen Dokumenten wie Bachelor-, Master- oder Doktorarbeiten ist es mehr als nur ratsam, sich um eine Strategie zur Sicherung der Daten kümmern. Dies gilt nicht nur für die Arbeit mit \LaTeX, sondern ganz allgemein. 

Einzelne Dateien oder ganze Festplatten/SSDs können kaputt gehen, Rechner gestohlen werden, zusammengefasst: externe Ereignisse können dazu führen, dass die Arbeit, in die man Wochen oder Monate gesteckt hat, nicht mehr verfügbar ist. 

Die Kosten für die professionelle Rettung von Daten von einer defekten Festplatte können leicht vierstellige Beträge erreichen, im Fall des Verlusts oder Diebstahls sind die Daten schlichtweg fort!

Die übliche Vorgehensweise, das Sichern von Dateien mit unterschiedlichen Zeitstempeln im Namen, ist besser als gar keine Sicherung zu haben, aber auch nicht unbedingt empfehlenswert. Sie schützt nur gegen Fehler, die man selbst im Dokument einbaut, nicht gegen Verlust oder technischen Defekt. Auch gegen einen Virus oder Verschlüsselungstrojaner, der alle Dateien auf dem Rechner löscht oder verschlüsselt und -- falls man Glück hat -- gegen anonyme Zahlung eines signifikanten Geldbetrages wieder entschlüsselt, hilft dies nicht.

In diesem Kapitel möchten wir verschiedene Möglichkeiten zeigen, wie man mit Hilfe eines Versionsverwaltungssystems Sicherungskopien seiner Arbeit verwalten kann.

\section{Grundlagen}

In der Softwareentwicklung setzt man seit Jahrzehnten auf die Nutzung von Versionsverwaltungssystem wie CVS/RSC, Subversion und Git, die -- neben dem Aspekt der Sicherungskopie außerhalb der eigenen vier Wände -- die Möglichkeit bieten, zu früheren Versionen einer Datei zurückzukehren. So könnte man beispielsweise zur letzten \TeX-Datei zurückgehen, die sich noch problemlos übersetzen ließ und diese mit der aktuellen, fehlerhaften Version vergleichen. Die meisten Betriebssysteme bringen zu diesem Zweck passende Werkzeuge mit, oftmals haben auch die Versionsverwaltungssysteme passendes an Bord.

In diesem Kapitel soll am Beispiel von Subversion und Git gezeigt werden, wie bei der Arbeit mit \TeX/\LaTeX\ Versionsverwaltungssysteme sinnvoll eingesetzt werden können.

Ein Hinweis noch: Die ordentliche Konfiguration einer Versionsverwaltung setzt ein gewisses technisches Verständnis voraus. Wer sich dies nicht zutraut, nutzt am besten einen der vielen Hoster, die die notwendige Infrastruktur und Expertise bereithalten. Github beispielsweise berechnet 7~US-\$ pro Monat für eine unbegrenzte Zahl privater (nicht öffentlich einsehbarer) Git-Repositories (die auch per Subversion-Client benutzt werden können), LCube-Hosting (\url{https://www.lcube-webhosting.de}) berechnet 2,99~Euro pro Monat für ein Gigabyte an Subversion-Repositories.

% Preis Dezember 2019: 2,49 €
% Vergleich: Dieses Buch < 100 MB.

\section{Subversion}

Subversion wurde ursprünglich von CollabNet entwickelt, seit 2009 hat es seine Heimat bei der Apache Foundation gefunden. Es handelt sich bei Subversion um eine \textit{zentrale} Versionsverwaltung, das bedeutet, das es einen zentralen Server gibt, auf dem der Subversion-Dienst läuft und auf den geänderte Dateien hochgeladen werden.

\subsection{Installation}

Es gibt verschiedene Möglichkeiten, einen Subversion-Server zu installieren. Die lokale Installation auf dem Arbeitsrechner -- egal auf welchem Betriebssystem --ist nur dann empfehlenswert, wenn die Daten regelmäßig, am besten auf einem externen Rechner, gesichert werden. Wenn nämlich der schlimmste Fall eintritt und der lokale defekt ist oder abhanden gekommen ist, ist nicht nur die Arbeitskopie, sondern auch das ganze Repository defekt.

Ein Hinweis noch zum Thema \enquote{Backup}: es empfiehlt sich, auch die Integrität des Backups gelegentlich zu prüfen.
Ein kaputtes Backup bietet keinen Mehrwert!

Der Autor selbst hat gute Erfahrungen mit der Nutzung eines Synology NAS (NAS = \enquote{Network-Attached Storage}), also einer Netzwerkfestplatte der Firma Synology gemacht, auf der der Subversion-Dienst mit wenigen Handgriffen installiert werden kann. 
Hier gilt zwar auch, dass eine externe Sicherung sehr ratsam ist, mit den bordeigenen Mitteln des NAS ist dies aber recht leicht umsetzbar. 
Der Autor hat es so gelöst, dass täglich die letzte Version eines jeden Repositories in eine Datei gedumpt wird (mittels \texttt{svnadmin dump}), die dann verschlüsselt und gezippt bei einem Cloud-Anbieter gespeichert wird.

Neben der Installation auf einem NAS kann man Subversion auch in Form eines Windows-Dienstes installieren, hier empfiehlt sich beispielsweise \enquote{VisualSVN Server} (\url{https://www.visualsvn.com/server/}), das in der Standard-Installation kostenlos verfügbar ist. Die Installation ist in wenigen Minuten erledigt, über das Programm \enquote{VisualSVN Server Manager} werden dann die User und Repositories konfiguriert. Für die folgenden Betrachtungen gehen wir davon aus, das a) VisualSVN auf dem lokalen Windows-Rechner läuft und b) ein leeres Repository \enquote{BachelorArbeit2017} eingerichtet wurde.

% https://localhost/!/#BachelorArbeit2017

\subsection{Subversion von der Kommandozeile}

Subversion bringt einen eigenen Kommandozeilen-Client mit, der auf allen Betriebssystemen identisch funktioniert. 
Schon allein aus diesem Grund empfiehlt es sich, wenigstens die Grundlagen dazu zu beherrschen.

Als erstes erfolgt der Checkout der Dateien aus dem \enquote{BachelorArbeit2017}-Repository in ein leeres Verzeichnis, welches das Arbeitsverzeichnis sein wird. 
Dazu wechselt man auf die Kommandozeile (über die \enquote{Eingabeaufforderung}) und wechselt mittels \texttt{cd} in das entsprechende Verzeichnis, wo das Arbeitsverzeichnis angelegt werden soll.

Dann gibt man den Befehl zum Checkout der Daten ein:

\texttt{svn checkout https://localhost/svn/BachelorArbeit2017/}

Nach der Eingabe von Benutzername und Passwort lädt Subversion dann die Dateien herunter. 
Da aber unser Repository noch neu und leer ist, erkennt man auf den Blick keinen Unterschied. 
Den Unterschied erkennt man nur dann, wenn man sich mittels \texttt{dir /ah} die versteckten Dateien und Verzeichnisse anzeigen lässt. 
Subversion hat beim Checkout der Daten das Verzeichnis \texttt{.svn} angelegt, in dem SVN-interne Daten gespeichert werden. 

Wenn wir jetzt im Arbeitsverzeichnis neue Ordner und Dateien anlegen, können diese mit \texttt{svn add <Dateiname/ordner>} unter die Versionverwaltung gestellt werden. 
Ein anschließendes \texttt{svn commit -m ''Kommentar''} überträgt die Dateien in das zentrale Repository.
Die Option \texttt{-m} mit dem Kommentar, was man denn geändert hat, bzw. die Option \texttt{-f}, die diesen Kommentar aus einer Datei liest, muss gesetzt sein. Der Kommentar kann auch leer sein, es gilt aber: je besser der Kommentar, desto leichter lassen sich möglicherweise auftretende Probleme später lösen.

Vor der Arbeit an den ausgecheckten Dateien empfiehlt es sich, jedes Mal ein Update der Dateien zu machen. Dazu nutzt man den Befehl \texttt{svn update}.  

\subsection{Grafische Clients}

Für die verschiedenen Betriebssysteme existieren auch diverse grafische Clients für Subversion.
Maß aller Dinge -- zumindest nach Meinung des Autors -- ist dabei aber TortoiseSVN (\url{https://tortoisesvn.net/downloads.html}). 
Auf dieses hilfreiche Werkzeug möchten wir daher näher eingehen. 
Nach Download, Installation und Neustart findet man im Kontextmenü des Explorers (rechte Maustaste \ldots)  zwei neue Menüpunkte: \texttt{SVN Checkout} und \texttt{TortoiseSVN}.

Über den Menüpunkt \texttt{SVN Checkout} können wir -- analog zum Kommandozeilenbefehl \texttt{svn checkout} Dateien aus einem SVN-Repository holen.
Wenn wir dies tun und dann in das neu erstellte Verzeichnis mit den ausgecheckten Dateien gehen, verändert sich das Kontextmenü. Der Punkt \texttt{SVN Checkout} ist nicht mehr verfügbar, was auch logisch ist, schließlich befinden wir uns bereits in einem ausgecheckten Verzeichnis. Hinzugekommen sind die Einträge \texttt{SVN Update} und \texttt{SVN Commit}. \texttt{SVN Update} tut genau das, was man erwarten würde: es aktualisiert die Dateien im Arbeitsverzeichnis mit dem letzten Stand aus dem Repository. Es empfiehlt sich daher, jedes Mal vor der Veränderung der lokalen Dateien solch ein \texttt{SVN Update} durchzuführen, um Datei-Konflikte zu vermeiden. Was in einem solchen Fall zu tun ist, werden wir noch später besprechen.

\texttt{SVN Commit} sorgt dafür, das die lokal veränderten Dateien wieder in das zentrale Repository hochgeladen werden. Man kann beim Commit via TortoiseSVN, genau wie auf der Kommandozeile, noch einen Kommentar mitgeben.

\section{Datei-Konflikte}

Früher oder später geschieht es immer: man hat lokal Dateien angepasst, die aber im Subversion-Repository bereits in veränderter Version vorliegen. Beim Commit der veränderten Dateien beschwert sich Subversion dann entsprechend.

\section{Integration von Subversion-Informationen in \LaTeX}

\subsection{svn-multi}


\section{Git}

Für dieses Buch wurde nicht Subversion, sondern git benutzt, das im Gegensatz zu Subversion nicht \textit{zentral}, sondern \textit{dezentral} arbeitet. 

Im wesentlichen bedeutet \textit{dezentral}, dass es nicht einen zentralen Server gibt, der die Historie vorhält, sondern dass jeder Entwickler oder Autor die komplette Historie auf dem Rechner hat und Änderungen an den lokalen Dateien auch in das lokale git Repository eingecheckt werden. 

Der Grund der Entscheidung für git statt Subversion war dabei eher organisatorisch als technisch, mehrere der Co-Autoren hatten bereits gute Erfahrungen mit github (\url{www.github.com}) gesammelt, das -- wie der Name sagt -- auf git aufbaut.

Man kann aber die lokalen Änderungen in ein entferntes (\enquote{remote}) Repository pushen, sodass es keine fundamentalen Unterschiede zur Arbeit mit git statt Subversion gibt.
Github unterstützt auch den Zugriff auf die Repositories über das Subversion-Protokoll, damit lassen sich auch grafische Clients wie TortoiseSVN mit github nutzen. 

Im folgenden soll gezeigt werden, wie man auf github ein neues Repository anlegt und damit Dateien verwaltet.


\endinput

Die aus Sicht des Autors beste Lösung ist daher eine Sicherung außerhalb des lokalen Rechners, idealerweise auch auf einem anderen Dateisystem. Der Grund hierfür ist technisch: Es sind Fälle bekannt, in denen Trojaner nicht nur die lokalen Dateien sondern auch alle Dateien auf angeschlossenen Netzlaufwerken verschlüsselten. Eine Sicherung auf einer Netzwerkfreigabe kann daher auch zu wenig sein. 



Und noch ein weiterer Aspekt soll genannt werden: die Arbeit mit mehreren Autoren und/oder auf mehreren Rechnern. Der Autor dieses Kapitels speichert wichtige Dokumente in einem Subversion-Repository. Soll mit einem anderem Rechner an den Dateien gearbeitet werden, so wird dort einfach ein neuer Checkout gemacht. Es darf nur am Ende der Arbeit nicht vergessen werden, die Dateien auch wieder mit dem Repository abzugleichen, dieser wichtige Schritt geht einem aber nach kurzer Zeit \enquote{in Fleisch und Blut} über.

Die Zusammenarbeit in einem Team vereinfacht sich drastisch, da der mühsame und fehlerträchtige Austausch von Dateien über FTP oder Email entfällt, ältere Versionen einer Datei können problemlos wiederhergestellt werden und die Zusammenführung verschiedener Versionen vereinfacht sich. Ein nicht zu unterschätzender Vorteil ist zudem die Möglichkeit, Backups quasi im \enquote{Vorbeigehen} zu erstellen.

