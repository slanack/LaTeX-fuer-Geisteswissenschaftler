% !TeX root = lfgw.tex
\chapter{Texte parallel setzen}
\autor{Philipp Pilhofer}

Das hier wird auf \paket{reledpar} umgestellt ...

\section{Spaltenweise}


%Das Paket \paket{parallel} von Matthias Eckermann stellt eine Umgebung 
%\lstinline/Parallel/ zur Verfügung, in der mit den Befehlen 
%\lstinline/\ParallelLText{...}/ und 
%\lstinline/\ParallelRText{...}/ die jeweils links- bzw. rechtsstehenden Texte
%angegeben werden können.
%
%Durch den Befehl \lstinline/\ParallelPar/ wird ein neuer \enquote{Doppelabsatz} begonnen.
%
%\begin{lstlisting}
% \linenumbers
% \begin{Parallel}{.45\textwidth}{.45\textwidth}
%  \ParallelLText{Gallia est omnis divisa in partes tres ...}
%  \ParallelRText{Gallien als ganzes ist in drei Teile...}
%  \ParallelPar
%  \ParallelLText{Hi omnes...}
%  \ParallelRText{Sie alle...}
% \end{Parallel}
%\end{lstlisting}
%
%Beim Erzeugen der \lstinline/Parallel/-Umgebung ist für die linke und rechte Seite ihre
%jeweilige Breite anzugeben; das kann als absolute Angabe (in mm oder cm) oder als 
%Bezugnahme zu einem Wert wie \lstinline/\textwidth/ (der Breite des aktuellen 
%Satzspiegels) erfolgen.
%
%Das obige Beispiel erzeugt folgende Ausgabe:
%\bigskip 
%
%\linenumbers
%\begin{Parallel}{.45\textwidth}{.45\textwidth}
%  \ParallelLText{Gallia est omnis divisa in partes tres, 
%    quarum unam incolunt Belgae,
%    aliam Aquitani,
%    tertiam, qui ipsorum lingua Celtae, nostra Galli appellantur.}
%  \ParallelRText{Gallien als ganzes ist in drei Teile gegliedert,
%    deren erster die Belger bewohnen,
%    den zweiten die Aquitanier,
%    den dritten jene, die in ihrer eigenen Sprache Kelten, in der unseren Gallier genannt werden.}
%  \ParallelPar
%  \ParallelLText{Hi omnes lingua institutis legibus inter se differunt.}
%  \ParallelRText{Sie alle unterscheiden sich hinsichtlich der Sprache, 
%    der (staatlichen) Einrichtungen und der Gesetze von einander.}
% \end{Parallel}
%\nolinenumbers
%
%(Zu der Zeilennummerierung vgl. Abschnitt \ref{zeilennummer} auf S.~\pageref{zeilennummer}.)
%
%\minisec{Das Paket \paket{paracol}}
%
%\paket{paracol}

\section{Seitenweise}

%\paket{ledpar}\footcite[175\psqq]{rouquette:2012}

