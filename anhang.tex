% !TeX root = lfgw.tex
\chapter{Anhang}

\section{Ein Beispiel, das (fast) alles kann}

In der folgenden Musterdatei wird (fast) alles vorgeführt, was das Skript erklärt.
Sie kompiliert in kile durch ALT+6...

\lstinputlisting{lfgw-musterdatei}


\section{Unicode}
\label{unicode} \index{Unicode}

\subsection{Einstellen des Editors auf Unicode}

\subsection{Umcodieren vorhandener Dateien}

Programm recode


\subsection{Häufig benötigte Unicode-Zeichen}

\label{utf8codes}

\section{Wie installiere ich die Software/Pakete etc.}

\minisec{Die radikale Alternative: Cloud-Lösung}
\index{overleaf}
\index{sharelatex}


\minisec{Windows: miktex}
\index{miktex}

\minisec{Standardwerkzeug der Linux-Distribution}

\minisec{ctan}
\index{ctan}


\section{Woher beziehe ich Dokumentation zu den Paketen?}

\minisec{Welche Pakete könnten interessant sein?}

ctan

\minisec{Wie funktionieren die schon installierten Pakete?}

texdoc PAKETNAME


\section{Welche Bücher sollte ich mir kaufen?}

Erster Schritt: Dokument \enquote{\LaTeXe -Kurzbeschreibung} mit 
\lstinline/texdoc lshort/. Dieses Dokument (ca. 50 Seiten) sollte man am besten ausdrucken.

Einen grundlegenden Überblick über das Gesamtsystem bietet
\cite{voss:einfuehrung}

Wenn man sich zur Benutzung der \KOMAScript -Klassen entscheidet, ist die ultimative Referenz,
mit der man erst das ganze Paket ausnutzen kann:
\cite{kohm:2014}

Die einzige Spezialmonographie zum Thema \LaTeX{} in den Geisteswissenschaften ist
\cite{rouquette:2012}

Ideal zum Nachschlagen bestimmter Befehle eignet sich:
\cite{voss:referenz}

Spezialtitel, je nach individuellem Bedürfnis:

\cite{voss:praesentationen}

\cite{voss:bibliografien}

\cite{voss:pstricks}


\section{Bücher veralten. Wer hält mich auf dem Laufenden?}

dante e.\,V.
