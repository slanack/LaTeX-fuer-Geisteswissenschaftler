\documentclass[a5paper]{article} 

\usepackage[pass]{geometry}

\usepackage[latin,ngerman]{babel}

\usepackage[%
series={},%keine Apparate aktivieren
noend,%keine Endotenapparate
noeledsec,%keine eledsections et al.
noledgroup%keine ledgroups
]{reledmac}%

\usepackage{reledpar}

\begin{document}

%%% die se\firstlinenumR{2}
\linenumincrementR{2}
\setlength{\columnrulewidth}{0.5pt}
\setlength{\Lcolwidth}{0.425\textwidth}
\setlength{\Rcolwidth}{0.425\textwidth}
\columnsposition{C}

\begin{pages}
\begin{Leftside}
\beginnumbering
\autopar
\selectlanguage{latin}\itshape

\pstart[\section*{Itinerarium Egeriae XXIII 1--4}]
\noindent 1. Nam proficiscens de Tharso perueni ad quandam ciuitatem supra mare adhuc Ciliciae, que appellatur Ponpeiopolim. Et inde iam ingressa fines Hisauriae mansi in ciuitate, quae appellatur Corico. Ac tertia die perueni ad ciuitatem, quae appellatur Seleucia Hisauriae. Ubi cum peruenissem, fui ad episcopum uere sanctum ex monacho, uidi etiam ibi ecclesiam ualde pulchram in eadem ciuitate.
\pend

\autopar

2. Et quoniam inde ad sanctam Teclam, qui locus est ultra ciuitatem in colle sed plano, habebat de ciuitate forsitan mille quingentos passus, malui ergo perexire illuc, ut statiua, quam factura eram, ibi facerem. Ibi autem ad sanctam ecclesiam nichil aliud est nisi monasteria sine numero uirorum ac mulierum.

3. Nam inueni ibi aliquam amicissimam michi, et cui omnes in oriente testimonium ferebant uitae ipsius, sancta diaconissa nomine Marthana, quam ego aput Ierusolimam noueram, ubi illa gratia orationis ascenderat; haec autem monasteria aputactitum seu uirginum regebat. Quae me cum uidisset, quad gaudium illius uel meum esse potuerit, nunquid uel scribere possum?

4. Sed ut redeam ad rem, monasteria ergo plurima sunt ibi per ipsum collem et in medio murus ingens, qui includet ecclesiam, in qua est martyrium, quod martyrium satis pulchrum est. Propterea autem murus missus est ad custodiendam ecclesiam propter Hisauros, quia satis mali sunt et frequenter latrunculantur, ne forte conentur aliquid facere circa monasterium, quod ibi est deputatum.

\endnumbering
\end{Leftside}

\begin{Rightside}
\beginnumbering

\pstart[\section*{Itinerarium Egeriae XXIII,1--4}]
...
\pend

\autopar

2

3

4

\endnumbering
\end{Rightside}
\end{pages}

% hier nun alles ausspucken, linke und rechte Seiten
\Pages

\end{document}