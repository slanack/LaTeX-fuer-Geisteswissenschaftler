\documentclass[ngerman]{dtk}
\usepackage[utf8]{inputenc}
\usepackage[T1]{fontenc}
%\usepackage[a4,center,cross]{crop}
\usepackage{url}
\usepackage{graphicx}


\begin{document}
	
	
\title{\LaTeX\ für Geisteswissenschaftler -- 
    Ein Projekt zur zielgruppenspezifischen Dokumentation von \TeX\ und co.}
\Author{Thomas Hilarius}{Meyer}%
	{Jahnstraße~7\\
	66453~Rubenheim\\
	\Email{thomas.hilarius.meyer@gmail.com}}
\maketitle


\begin{abstract}
Seit November 2016 arbeitet eine kleine Arbeitsgruppe 
-- Lukas C. Bossert, Craig Parker-Feldmann, Martin Sievers, Thomas H. Meyer, Philipp Pilhofer und Uwe Ziegenhagen --
an der Erstellung eines Einführungsbuches zu \LaTeX ,
das sich besonders an den spezifischen Bedürfnissen von Geisteswissenschaftlern orientiert.
Was sind diese besonderen Bedürfnisse?
Wie ist der aktuelle Stand des Projektes?
In welchen Bereichen wird noch Unterstützung benötigt?
Ein Überblick\ldots 
\end{abstract}

\section{Wie kam es dazu?}

Auf der vereinseigenen Mailingliste flammt immer wieder das Thema auf, ob die Gemeinde der \LaTeX\-Nutzer
sich stärker um das Thema \enquote{Öffentlichkeitsarbeit} bemühen müsste.
Dabei stellte sich heraus, dass es sich bei \TeX\ und co. um Minderheiten- und Spartensoftware handelt, die
wahrscheinlich immer nur von einer Minderheit aller Computernutzer eingesetzt werden.
Dabei ist ausschlaggebend, dass \LaTeX\ eine Reihe von Möglichkeiten bietet, die den nicht ganz leichten Einstieg
und Lernaufwand rechtfertigen. 
Im Bereich der Mathematik und Naturwissenschaften ist dieses Potential weitgehend bekannt und die Software
dementsprechend eher verbreitet.
Allerdings hat die \TeX -Welt auch Vertretern der Geisteswissenschaften einiges zu bieten, was sich mit 
Mainstream-Programmen nur mit sehr viel Handarbeit und unter deutlichen Qualitätseinbußen bewerkstelligen lässt.

Allerdings ist \TeX\ im Bereich der Geisteswissenschaften \emph{noch} weniger verbreitet, als im Bereich der
sog. \emph{hard sciences}.

Dies liegt m.E. auch daran, dass es an einer Übersichtsdarstellung fehlt, die dem mit einer gewissen 
computerischen Offenheit ausgestatteten Philologen, Historiker oder Theologin deutlich machen würde,
welche Möglichkeiten die Softwarewelt um \LaTeX\ zu bieten hat. 
Gleichzeitig sollte die Publikation einen gangbaren Weg zeigen, wie sich die jeweiligen Wünsche verwirklichen
lassen. Wer dann Blut geleckt hat und spezifischere Sonderwünsche verwirklichen will, kann sich dann anhand
der meistens vorzüglichen Paketdokumentationen weiter einarbeiten. 
Aus dem gewünschten Einführungsskript wäre er dann herausgewachsen\ldots 

Im ersten Eifer habe ich eine Gliederung zusammengestellt und auch begonnen, Einzelpunkte auszuführen.
Dabei ist mir aufgefallen \emph{wie} viele Möglichkeiten \LaTeX\ bietet -- und auf wievielen Gebieten ich nichts
zu sagen hätte.

An diesem Punkt ist die Diskussion über Öffentlichkeitsarbeit auf der Dante-ev-Mailingliste wieder aufgekommen
und ich habe meine Projekt-bau-ruine öffentlich angesprochen. Und siehe da: Innerhalb von wenigen Tagen
ist ein hoch motiviertes Team zusammengekommen, noch viel mehr \TeX nische Möglichkeiten und ein
massiver Motivationsschub.


\section{Was ist geplant?}

Zielvorstellung ist im Moment ein Handbuch von überschaubarem Umfang (ca. 240 Seiten?), das evtl. in der 
Dante-Reihe erscheinen könnte.

Beim Leser werden keine \LaTeX -Kenntnisse vorausgesetzt; als mögliche Zielgruppe kommen zwei Arten von Lesern
in Betracht:

\begin{itemize}
    \item (Künftige) \LaTeX -Nutzer mit geisteswissenschaftlichem Hintergrund, die sich zunächst dafür
         interessieren, ob sich \LaTeX\ für ihre spezifischen Bedürfnisse und Anforderungen eignet und
         wie sie diese mit einfachen Mitteln und überschaubarem Aufwand befriedigen können.
    \item Erfahrene \TeX niker, die in geisteswissenschaftlichen Projektzusammenhängen ihre Hilfe angeboten
         haben und jetzt etwas genauer wissen möchten, was auf sie zukommt -- und wie sie die in sie
         gesetzten Hoffnungen und Erwartungen erfüllen können.
\end{itemize}

Diese Zielsetzung bestimmt Breite und Tiefe der Darstellung: 

Die Bandbreite der behandelten Themen und Lösungsmöglichkeiten soll möglichst groß sein.
Mit den skizzierten Paketen sollte es möglich sein, eine kritische Ausgabe eines griechischen Textes in Versen mit
arabischen Glossen durch ein rückläufig sortiertes Reimverzeichnis zu erschließen.
Wer nur Teilmengen davon realisieren muss, hat Glück gehabt\ldots

Hinsichtlich der Tiefe der Behandlung der verschiedenen Klassen und Pakete ist klar, dass mehr als ein
ganz oberflächlicher Blick nicht realisiert werden kann. Das ist auch nicht nötig, da in allen Fällen bereits
gute Paketdokumentationen existieren, die bei Spezialproblemen konsultiert werden können -- wenn man erstmal
weiß, wo man suchen muss.

Im folgenden werden die Einzelpunkte, die in dem Einführungsshandbuch \enquote{\LaTeX\ für Geisteswissenschaftler}
erklärt werden sollen, kurz vorgestellt.


\section{Was sind die Themen?}


\minisec{Kap. 1 -- Grundsätzliches}

Neben einer kurzen Geschichte von \TeX\ und co. geht es um die Frage der Installation sowie die Benutzung 
geeigneter Editoren.

\minisec{Kap. 2 -- Texte schreiben}

Im ersten Schritt wird gezeigt, wie mit \LaTeX\ wissenschaftliche Texte mit ihren zahlreichen Bestandteilen
verfassen kann: mit Textgliederungen, Schriftauszeichnungen, besonderen Schriftarten (z.B. Fraktur- und
Schreibschriften), Sonderzeichen, Listen und Aufzählungen, Zitaten, Fußnoten, Marginalien, mehrspaltigen
Einschüben, Grafiken und Gleitumgebungen, Tabellen, einheitlich formatierten Bibelstellen (vgl. den Aufsatz
in diesem Heft), Querverweisen, Einschüben mit Lyrik oder ganze Dramen.  

\minisec{Kap. 3 -- Diagramme zeichnen}

Hier geht es darum, die Möglichkeiten von TikZ und PSTricks für spezifisch geisteswissenschaftliche
Bedürfnisse anzuwenden: Zeitschienen, Handschriften-Stemmata, Stammbäume, linguistische Strukturen etc.
(Hier wäre wir noch für fachkundige Hilfe dankbar!)

\minisec{Kap. 4 -- Textpassagen in nicht-lateinischen Alphabeten einbetten}

Die Zahl der mit \LaTeX\ verarbeitbaren Schriftsysteme, die im Bereich der Geisteswissenschaften von besonderem 
Interesse sind, ist enorm:
Griechisch,
Hebräisch,
Russisch,
Koptisch,
Altkirchenslawisch,
Arabisch,
Texte in Hieroglyphen,
Keilschriften,
Runen,
phonetischen Alphabete,
verschiedene Kurzschriften\ldots

(Auch hier wären wir noch für kompetente Mitarbeiter, die mit den Sprachpaketen wirklich produktiv 
arbeiten, dankbar. Die Autoren haben nur für Griechisch und Hebräisch Erfahrung.)

\minisec{Kap. 5 -- Zusammenhängende Texte parallelisieren}

Hier geht es v.a. um das Nebeneinanderstellen von Original und Übersetzung auf verschiedenen Ebenen:
als Interlinearglosse oder spalten- bzw. seitenweise.

\minisec{Kap. 6 -- Kritische Apparate setzen}

Die Erstellung kritischer Apparate mit \Paket{Reledmac} von Ma\"{\i}eul Roquette allein kann schon den
Einsatz von \LaTeX\ rechtfertigen. (Einziger dem Autor bekannter Konkurrent für diese \enquote{Königsklasse} 
geisteswissenschaftlichen Computereinsatzes ist das wesentlich schwerer
zugängliche TUSTEP vom Zentrum für Datenverarbeitung der Universität Tübingen.)

\minisec{Kap. 7 -- Literatur und Zitate automatisch verwalten}

Auch die Übertragung der Zitatverwaltung an \Paket{biblatex} spart nach erfolgter Einarbeitung eine erhebliche
Menge Lebenszeit. Es existieren eine Reihe von Zitierstile speziell für geisteswissenschaftliche Bedürfnisse.

\minisec{Kap. 8 -- Texte durch Register erschließen}

Neben den üblichen Anforderungen an den alphabetischen Index sowie der Möglichkeit, mehrere Register
automatisch zu erstellen, geht es auch um Spezialregister, z.B. einen Bibelstellenindex.

\minisec{Kap. 9 -- Prüfungen erstellen mit \Paket{exam}}

Die Erstellung von Prüfungen mithilfe des Paketes \Paket{exam} bietet auch im geisteswissenschaftlichen
Kontext Vorzüge.

\minisec{Kap. 10 -- Präsentationen gestalten mit \Paket{beamer}}

Dasselbe gilt für die Benutzung von \Paket{beamer}, das -- etwas Fingerfertigkeit mit \LaTeX\ vorausgesetzt --
Präsentationen effizienter gestalten lässt, als Powerpoint.

\minisec{Kap. 11 -- Versionen verwalten mit Subversion und Git}

Geisteswissenschaftliche Projekte als Teamwork können -- ebenso wie Programmierprojekte -- erheblich von
der Benutzung von Versionsverwaltungssystemen profitieren.
In diesem Kontext ist es besonders von Vorteil, dass \LaTeX\ mit bloßen Textdateien arbeitet, die die
Versionierung erheblich besser ermöglichen, als geschlossene Dateiformate wie bei Word.

\minisec{Kap. 12 -- Einbetten von \LaTeX\ in heterogene Projekte}

\LaTeX\ steht nicht allein in der Welt, sondern lässt sich wundervoll z.B. mit Skriptsprachen wie Perl oder
Python kombinieren. Und mit Lua steht in Lua\LaTeX\ eine eigene eingebaute Skriptsprache zur Verfügung.
(Um deren Vorzüge überzeugend herauszuarbeiten, wären wir ebenfalls für kompetente Mithilfe dankbar.
Wie lassen sich Dinge wie Worthäufigkeitslisten oder rückwärtig sortierte Indices m.H. von Lua elegant
realisieren?)


\section{Wie kann ich helfen?}

Die Mitglieder des \enquote{\LaTeX\ für Geisteswissenschaftler}-Projektes wären für zahlreiche Formen von
Hilfen dankbar:

\begin{itemize}
    \item Vielleicht kennen Sie sich mit einem bestimmten Paket besonders gut aus und können seine Bedeutung
        für geisteswissenschaftliche Projekte erklären?
    \item Vielleicht ist Lua Ihre zweite Muttersprache und Sie könnten bestimmte \enquote{Wünsche} erfüllen?
    \item Vielleicht vermissen Sie in dieser Übersicht ein Thema, das sie für wichtig halten?
\end{itemize} 

In allen diesen -- und vielen anderen Fällen -- sollten Sie mit uns Kontakt aufnehmen\ldots

Das Projekt ist zugänglich unter \url{http://github.com/thomas-hilarius-meyer/latex-fuer-geisteswissenschaftler}

\end{document}