% !TEX root = lfgw.tex

\RequirePackage{iftex}
\RequireLuaTeX
\RequirePackage{luatex85}

\documentclass[%
   ibycus,polutonikogreek,english,french,latin,ngerman,%global definiert!
   %%%draft=false,%
   fontsize=11pt,%
   paper=17cm:24cm,%
   DIV=13,%
   listof=totoc,%
   bibliography=totoc,%
   pagesize%
   ]{scrbook}

\usepackage{yfonts}
\usepackage{textgreek}
\usepackage{cjhebrew}[2017/03/06]
\usepackage{amsmath,amssymb}
%%%\usepackage{textcomp}
\usepackage{unicode-math}
\usepackage{fontspec}
\defaultfontfeatures{Ligatures=TeX,Scale=MatchLowercase}
%\newfontfamily\GFSDidot{GFS Didot}
\newcommand\gkk[1]{%{\GFSDidot
#1}

\usepackage[]{babel}
\usepackage[ngerman,noftligs]{selnolig}
%%%\setmainfont{Libertinus Serif} %reicht auch, unten evtl. besser wegen SB
\setmainfont{libertinusserif}[
   Extension = {.otf},
   UprightFont = {*-regular}, ItalicFont = {*-italic},
   BoldFont = {*-bold}, BoldItalicFont = {*-bolditalic},%
   FontFace = {sb}{\updefault}{*-semibold},%
   FontFace = {sb}{it}{*-semibolditalic}]%
\setsansfont{Libertinus Sans}
\setmonofont[Scale=MatchLowercase,FakeStretch=0.85]{DejaVu Sans}%das wird für Griechisch in Codebeispielen gebraucht
\setmathfont{Libertinus Math}
\makeatletter
\newcommand*\sustyle{\addfontfeatures{VerticalPosition=Superior}}
\DeclareTextFontCommand{\textsu}{\sustyle}
\def\@makefnmark{\hbox{\sustyle\@thefnmark}}
\makeatother
\DeclareTextCommandDefault{\textborn}{{\char"002A}}
\DeclareTextCommandDefault{\textdied}{{\char"2020}}
\usepackage[% microtype
   final,%
   tracking=smallcaps,%
   expansion=alltext,%
   protrusion=true%
   ]{microtype}%
\SetTracking{encoding=*,shape=sc}{50}%
\UseMicrotypeSet[protrusion]{basicmath} % disable protrusion for tt fonts

\usepackage[headsepline]{scrlayer-scrpage}
\clearscrheadfoot
\ihead{\headmark}
\ohead{\pagemark}
\pagestyle{scrheadings}

\usepackage[autostyle]{csquotes}
\usepackage[newcommands]{ragged2e}

\usepackage{graphicx}
\graphicspath{{bilder/}}

\usepackage{grffile}
\usepackage[a4,center,cross]{crop}

\usepackage{listings}
\lstdefinestyle{listinglfgw}{%
  inputencoding=utf8,
  extendedchars=true,
  language=[LaTeX]{TeX},
  numbers=left, 
  %stepnumber=3,
  numbersep=5pt, 
  numberfirstline=false,  
  numberstyle=\tiny\textsf,
  basicstyle=\ttfamily\footnotesize,
  keywordstyle=\bfseries,%
  texcsstyle=*\ttfamily\footnotesize\bfseries,%
  %frame=tlrb,
  breaklines=true,
  breakatwhitespace=true,
  breakindent=5pt,
  %postbreak=\mbox{$\hookrightarrow$},
  escapeinside={*@}{@*},
  %showstringspace=false, 
  captionpos=b,
  upquote=true,
  %%%classoffset=0,
  morekeywords={%
  %a
  addbibresource,addplot,Afootnote,alteSeite,autocite,autopar,
  %b
  beginnumbering,Bfootnote,biblerefformat,biblerefmap,biblerefstyle,bonuspointpoints,
  %c
  Cfootnote,chapter,chbpword,chpword,chpgword,chqword,chsword,chtword,citeauthor,cites,citetitle,columnrulewidth,Columns,columnsposition,
  %d
  draw,
  %e
  edtext,endnumbering,enquote,
  %f
  firstlinenum,firstlinenumR,firstpageheadrule,footcite,footcites,footfullcite,fullcite,fillwithgrid,fillwithlines,fillwithdottedlines,Forest,Forest*,
  %h
  hpword,hqword,hsword,htword,hypertarget,hyperlink,
  %i
  includegraphics,includeonlyframes,
  %l
  Lcolwidth,ledsidenote,lemma,linenumincrement,linenumincrementR,
  %m
  msdata,mode,maketitle,
  %n
  newfontface,newfontfamily,node,
  %o
  only,onslide,
  %p
  Pages,parencite,parencites,part,pend,pibibleverse,pbibleverse,pointpoints,printbibheading,printbibliography,printindex,pstart,pgfimage,pgfusepath,
  pgfplothandlerlineto,pgfplotsset,pgfplotxyfile,
  %q
  quell,question,
  %r
  Rcolwidth,
  %s
 selectlanguage,setlength,setmainfont,setmonofont,setmsdatalabel,setromanfont,setsansfont,smartcite,smartcites,solutiontitle,setbeamerfont,setbeamercovered,subsection,subsection*,subsubsection,subtitle,
  %t
  text,textborn,textcite,textcites,textdelta,textDelta,textdied,texteuro,textgamma,textGamma,textmarried,textsubscript,thealteSeite,tableofcontents,th,TH,
  textalpha, textbeta, textgamma, textdelta, textepsilon, textzeta, texteta, texttheta, textiota, textkappa, textlambda, textmu, textnu,
  textxi, textomikron, textpi, textrho, textsigma, texttau, textupsilon, textphi, textchi,  textpsi, textomega,
  textAlpha, textBeta, textGamma, textDelta, textEpsilon, textZeta, textEta, textTheta, textIota, textKappa, textLambda, textMu, textNu,
  textXi, textOmikron, textPi, textRho, textSigma, textTau, textUpsilon, textPhi, textChi,  textPsi, textOmega,
  %u
  usetheme,usecolortheme,usebackgroundtemplate,useforestlibrary,usepackage,usetikzlibrary,
  %v
  vari,
  %x
  Xafternumber,Xarrangement,Xlemmaseparator,Xinplaceoflemmaseparator,Xinplaceofnumber,Xnonbreakableafternumber,Xnotenumfont,Xnumberonlyfirstinline,Xnumberonlyfirstintwolines,Xsymlinenum,Xtwolines,Xtwolinesbutnotmore,Xtxtbeforenotes
  },   
  literate=
  {á}{{\'a}}1 {é}{{\'e}}1 {í}{{\'i}}1 {ó}{{\'o}}1 {ú}{{\'u}}1
  {Á}{{\'A}}1 {É}{{\'E}}1 {Í}{{\'I}}1 {Ó}{{\'O}}1 {Ú}{{\'U}}1
  {à}{{\`a}}1 {è}{{\`e}}1 {ì}{{\`i}}1 {ò}{{\`o}}1 {ù}{{\`u}}1
  {À}{{\`A}}1 {È}{{\'E}}1 {Ì}{{\`I}}1 {Ò}{{\`O}}1 {Ù}{{\`U}}1
  {ä}{{\"a}}1 {ë}{{\"e}}1 {ï}{{\"i}}1 {ö}{{\"o}}1 {ü}{{\"u}}1
  {Ä}{{\"A}}1 {Ë}{{\"E}}1 {Ï}{{\"I}}1 {Ö}{{\"O}}1 {Ü}{{\"U}}1
  {â}{{\^a}}1 {ê}{{\^e}}1 {î}{{\^i}}1 {ô}{{\^o}}1 {û}{{\^u}}1
  {Â}{{\^A}}1 {Ê}{{\^E}}1 {Î}{{\^I}}1 {Ô}{{\^O}}1 {Û}{{\^U}}1
  {œ}{{\oe}}1 {Œ}{{\OE}}1 {æ}{{\ae}}1 {Æ}{{\AE}}1 {ß}{{\ss}}1
  {ç}{{\c c}}1 {Ç}{{\c C}}1 {ø}{{\o}}1 {å}{{\r a}}1 {Å}{{\r A}}1
  {€}{{\EUR}}1 {£}{{\pounds}}1
}
\lstset{style=listinglfgw}
  
\usepackage[% tcolorbox
  skins,%
  listings,%
  breakable,%
]{tcolorbox}
\tcbset{%
lfgwstyle/.style={%
    before skip=\baselineskip,
    boxrule=0pt,
    %bottomrule=2pt,
    %toprule=2pt,
    %colframe=black,
    colback=black!5,
    coltitle=white,
    bicolor,
    sharp corners,
    colbacklower=white,
    fonttitle=\sffamily\bfseries,
    breakable,
    %label=#1,
}}


\newtcblisting[auto counter,number within=chapter]{lfgwexample}[1]{%
    lfgwstyle,
    fontupper=\small\ttfamily,
    sidebyside,
    listing and text,
    title={Beispiel \thetcbcounter},
    listing options={style=listinglfgw},
    #1,
%text and listing,
}

\newtcblisting[use counter from=lfgwexample,number within=chapter]{lfgwcode}[1]{%
    lfgwstyle,
    fontupper=\small\ttfamily,
    listing only,
    title={Beispiel \thetcbcounter},
    listing options={style=listinglfgw},
    #1,
}

\newtcblisting[use counter from=lfgwexample,number within=chapter]{lfgwprint}[1]{%
    fontupper=\small,
    lfgwstyle,
    text only,
    title={Beispiel \thetcbcounter},
    listing options={style=listinglfgw},
    #1,
}


\usepackage{showexpl}
%\lstset{explpreset={escapeinside={*@}{@*}}}

\usepackage[pagewise]{lineno}
\usepackage{sidenotes}
\usepackage{multicol}
% Für den Abschnitt über Konstituentenanylyse von C. Römer:
\usepackage[linguistics]{forest}
\forestapplylibrarydefaults{linguistics,edges}
\useforestlibrary{edges}
\makeatletter
\let\pgfmathModX=\pgfmathMod@
\usepackage{pgfplots}%
\pgfplotsset{compat=1.14}
\let\pgfmathMod@=\pgfmathModX
\makeatother
%http://tex.stackexchange.com/questions/328972/presence-of-pgfplots-package-breaks-forest-environment-%w-folder-option-en/329015

% Pakete im Bereich Kapitel 3 - Diagramme zeichnen
\usepackage{tikz}
\usepackage[all]{genealogytree}
\usepackage{chronology}
\usepackage{pgf-pie}
\usetikzlibrary{pgfplots.dateplot}
\usetikzlibrary{mindmap}

\deffootnote[1.5em]{1.5em}{1.5em}{\makebox[1.5em][l]{{\fontseries{sb}\selectfont\thefootnotemark\ }}}

\usepackage{enumitem}			% for simple list modifications
\setlist{leftmargin=*,before=\setlength{\rightmargin}{\leftmargin}}

\usepackage{parallel}


%- Für Lyrik-Satz, Christine Römer:
\usepackage{verse}
\newcommand\He{\textbackslash\xspace}
\newcommand\Se{$\cup$\xspace} 
\newcommand*{\Zr}{\unskip\textunderscore}
\newcommand\daktylus{\textbackslash$\cup\cup$\xspace} 
\newcommand\anapaest{$\cup\cup$\textbackslash\xspace}

% ------------
\usepackage{runic}
\usepackage{hieroglf}
\usepackage[normalem]{ulem} %%% MS: Wofür? Unterstreichungen sollten vermieden werden! LCB: kann \xout
\usepackage{letterspace} %%% MS: Wofür. Besser direkt \textls aus dem Microtypepaket
\usepackage[hyphens]{url}
\usepackage{hologo}
\usepackage{philex}
\def\fg{}
\usepackage{siunitx} %Supreme typesetting of units
\sisetup{%
    tight-spacing=true, %
    %math-rm=\mathsf, 
    %		text-rm=\sffamily,
    detect-all, %Zahlen werden in der aktuellen Schrift angezeigt
    detect-family,
    exponent-to-prefix  	= true,%
    round-mode          	= places,% 
    round-precision     	= 2,%
    group-minimum-digits 	= 4, % Für "Tausenderpunkt" --> 1.234 anstatt 1234
    group-separator			={.},% für "12.345" statt "12 345"
    %  scientific-notation = engineering, % Use multiples of 3 as exponent
    locale					=DE, % Typeset numbers and units the German way
    range-phrase 			={$\times$},%
    %		zero-decimal-to-integer,%aus "2.0" wird "2"
    range-units				=single,  % --> 2 x 2 m, - auskommentieren für 2 m x 2 m
    %%%---------
%    unit-color=myred,
}

\usepackage[german]{keystroke}% Computer-Tastatur-Tasten in Anweisungs-Texte zu setzen (z.B. \Shift, \Ctrl, \Spacebar).

\usepackage{imakeidx}
\indexsetup{level = \subsection*, toclevel = subsection, noclearpage, headers = {\indexname}{\indexname}}
\makeindex[                title = {Allgemeiner Index}]
\makeindex[name = pakete,  title = {Verzeichnis der Paketnamen}]
\usepackage[% biblatex
   style=authoryear,  
%  style=historische-zeitschrift, % das am liebsten - aber der geht (bei mir?) nicht!
%  pageref=true,
	backend=biber
	]{biblatex}
\addbibresource{lfgw-bibliographie.bib}
%%%\defbibheading{bibliography}{\addchap{#1}} %%%Besser bei \addchap bleiben

\usepackage{fancyvrb}
\makeatletter
\usepackage[% reledmac
   series={A,B,C},%nur die Apparate A B C aktivieren
   noend,%keine Endotenapparate
   noeledsec%keine eledsections et al.
   %noledgroup%keine ledgroups -- das muss hier auskommentiert werden, damit die minipages funktionieren!
]{reledmac}%
\@ifpackagelater{reledmac}{2017/03/20}{%
   % Package is new enough
}{%
\PackageError{reledmac}{Es wird reledmac >= 2.18.1 benötigt.}%
}
\makeatother
\usepackage{reledpar}

\usepackage{varioref}
\usepackage{hyperxmp}
\usepackage{hyperref}
\hypersetup{					% setup the hyperref-package options
	pdftitle={LaTeX für Geisteswissenschaftler},	% 	- title (PDF meta)
	pdfsubject={Handbuch},% 	- subject (PDF meta)
	pdfauthor={varia},	% 	- author (PDF meta)
	pdfauthortitle={},
	pdfcopyright={Copyright (c) \the\year\. All rights reserved.},
	pdfhighlight=/N,
	pdfdisplaydoctitle=true,
	pdfdate={\the\year-\the\month-\the\day}
	pdflang={de},
   pdfencoding=unicode,   % Sorgt für korrekte Umlaute in den pdf-Lesezeichen - thm
	pdfcaptionwriter={varia},
	pdfkeywords={{LaTeX}, {Geisteswissenschaften}},
	pdfproducer={LuaLaTeX},
	pdflicenseurl={http://creativecommons.org/licenses/by-nc-nd/4.0/},
	plainpages=false,			% 	- 
   colorlinks   = true, %Colours links instead of ugly boxes
   urlcolor     =  blue!50!black, %Colour for external hyperlinks
   linkcolor    = blue, %Colour of internal links
   citecolor   = green!50!black, %Colour of citations
   linktoc=page,
  	pdfborder={0 0 0},			% 	-
	breaklinks=true,			% 	- allow line break inside links
 %   bookmarks=true,
	bookmarksnumbered=true,		%
	bookmarksopenlevel=2,
	bookmarksopen=true,		%
   bookmarksdepth=3,
   pdfdisplaydoctitle,
	final=true	% = true, nur bei web-Dokument!! (wichtig!!)
}
\usepackage{bookmark}%advanced bookmarks
\usepackage[% cleceref
  sort,
  nameinlink
  ]{cleveref}%nach hyperref laden

\crefname{tcb@cnt@lfgwexample}{Beispiel}{Beispiele}
\addto\captionsngerman{%
    \crefformat{lfgwexample}{#2Beispiel\,#1#3}%
    \crefformat{lfgwcode}{#2Beispiel\,#1#3}%
    \crefformat{lfgwprint}{#2Beispiel\,#1#3}%
}

\setkomafont{pagehead}{\normalcolor\normalfont\small\upshape}
\setkomafont{pagenumber}{\normalcolor\normalfont\normalsize\bfseries}

\renewcommand*{\glqq}{\textquotedblleft}
\renewcommand*{\grqq}{\quotedblbase}

\providecommand*{\reledmac}{\mbox{\Package{reledmac}}\xspace}

\providecommand*{\LaTeXTeX}{\hologo{(La)TeX}}
\providecommand*{\AmSLaTeX}{\hologo{AmSLaTeX}}
\providecommand*{\AmSTeX}{\hologo{AmSTeX}}
\providecommand*{\biber}{\hologo{biber}}
\providecommand*{\BibTeX}{\hologo{BibTeX}}
\providecommand*{\BibTeXacht}{\hologo{BibTeX8}}
\providecommand*{\ConTeXt}{\hologo{ConTeXt}}
\let\context\ConTeXt
\providecommand*{\emTeX}{\hologo{emTeX}}
\providecommand*{\eTeX}{\hologo{eTeX}}
\providecommand*{\ExTeX}{\hologo{ExTeX}}
\providecommand*{\HanTheThanh}{\hologo{HanTheThanh}}
\providecommand*{\iniTeX}{\hologo{iniTeX}}
\providecommand*{\KOMAScript}{\hologo{KOMAScript}}
\providecommand*{\LaTeX}{\hologo{LaTeX}}
\providecommand*{\LaTeXe}{\hologo{LaTeX2e}}
\providecommand*{\LaTeXIII}{\hologo{LaTeX3}}
\providecommand*{\LaTeXML}{\hologo{LaTeXML}}
\providecommand*{\LuaLaTeX}{\hologo{LuaLaTeX}}
\let\lualatex\LuaLaTeX
\providecommand*{\LuaTeX}{\hologo{LuaTeX}}
\let\luatex\LuaTeX
\providecommand*{\LyX}{\hologo{LyX}}
\providecommand*{\METAFONT}{\hologo{METAFONT}}
\let\MF\METAFONT
\providecommand*{\MetaFun}{\hologo{MetaFun}}
\providecommand*{\METAPOST}{\hologo{METAPOST}}
\providecommand*{\MetaPost}{\hologo{MetaPost}}
\let\MP\METAPOST
\providecommand*{\MiKTeX}{\hologo{MiKTeX}}
\providecommand*{\NTS}{\hologo{NTS}}
\providecommand*{\OzMF}{\hologo{OzMF}}
\providecommand*{\OzMP}{\hologo{OzMP}}
\providecommand*{\OzTeX}{\hologo{OzTeX}}
\providecommand*{\OzTtH}{\hologo{OzTth}}
\providecommand*{\PCTeX}{\hologo{PCTeX}}
\providecommand*{\pdfTeX}{\hologo{pdfTeX}}
\let\pdftex\pdfTeX
\providecommand*{\pdfLaTeX}{\hologo{pdfLaTeX}}
\let\pdflatex\pdfLaTeX
\providecommand*{\PiC}{\hologo{PiC}}
\providecommand*{\PiCTeX}{\hologo{PiCTeX}}
\providecommand*{\plainTeX}{\hologo{plainTeX}}
\providecommand*{\SageTeX}{\hologo{SageTeX}}
\providecommand*{\SLiTeX}{\hologo{SLiTeX}}
\providecommand*{\teTeX}{\hologo{teTeX}}
\providecommand*{\TeXivht}{\hologo{TeX4ht}}
\providecommand*{\TTH}{\hologo{TTH}}
\providecommand*{\virTeX}{\hologo{virTeX}}
\providecommand*{\VTeX}{\hologo{VTeX}}
\providecommand*{\XeLaTeX}{\hologo{XeLaTeX}}
\providecommand*{\XeTeX}{\hologo{XeTeX}}
%%
\newcommand\BibTool{\textsc{Bib\hskip-.1em
      T\hskip-.15emo\hskip-.05emo\hskip-.05eml}\xspace}
\providecommand*{\TikZ}{\textsf{Ti\textit{k}Z}}
%\providecommand*{\pgf/tikz}{\textsf{pgf/Ti\textit{k}Z}}
\def\pgf/tikz{\textsf{pgf/Ti\textit{k}Z}}
\providecommand*{\ALEPH}{\ensuremath{\aleph}}

\providecommand\eV{e.V\kern-0.18em\@ifnextchar.{}{.}\kern0.18em}
\providecommand\dante{\mbox{DANTE~\eV}}
\providecommand\Dante{DANTE,
   Deutschsprachige Anwendervereinigung \TeX~\eV}
\providecommand\DTK{Die \TeX\-ni\-sche Ko\-m{\"o}\-die}
\providecommand\PS{Post\-Script}
\providecommand\TUG{\TeX{} Users Group}
\providecommand\TUGboat{\textsl{TUGboat}}
\let\DANTE\dantelogo
\providecommand*{\TeXLive}{\TeX{}Live}

\def\BibLaTeX{Bib\hologo{LaTeX}}
\let\biblatex\BibLaTeX
\providecommand*{\CTAN}{\texttt{CTAN}\xspace}

\providecommand*{\prog}[1]{\texttt{#1}}
\let\Program\prog

\providerobustcmd*{\paket}[2][]{\textsf{#2}\index[pakete]{\if$#1$#2\else#1\fi}}
\let\Package\paket%zwecks kompatibilität mit DTK
\let\Paket\paket
\newcommand*{\opt}[1]{\texttt{#1}}
\newcommand*{\file}[1]{\texttt{#1}}
\newcommand*{\env}[1]{\texttt{#1}}

\makeatletter
\DeclareRobustCommand\cs[1]{\texttt{\bfseries\char`\\#1}}
\DeclareRobustCommand\meta[1]{%
   \ensuremath\langle
   \ifmmode \expandafter \nfss@text \fi
   {%
      \meta@font@select
      \edef\meta@hyphen@restore
      {\hyphenchar\the\font\the\hyphenchar\font}%
      \hyphenchar\font\m@ne
      \language\l@nohyphenation
      #1\/%
      \meta@hyphen@restore
   }\ensuremath\rangle
}
\def\marg{\@ifstar{\@@marg}{\@marg}}
\providecommand\@marg[1]{%
   {\ttfamily\mdseries\char`\{}\meta{#1}{\ttfamily\mdseries\char`\}}}
\providecommand\@@marg[1]{%
   {\ttfamily\mdseries\char`\{}{\mdseries #1}{\ttfamily\mdseries\char`\}}}
\def\oarg{\@ifstar{\@@oarg}{\@oarg}}
\providecommand\@oarg[1]{%
   {\ttfamily\mdseries[}\meta{#1}{\ttfamily\mdseries]}}
\providecommand\@@oarg[1]{%
   {\ttfamily\mdseries[}{#1}{\ttfamily\mdseries]}}
\providecommand\parg[1]{%
   {\ttfamily\mdseries(}\meta{#1}{\ttfamily\mdseries)}}
\def\meta@font@select{\itshape\mdseries}
\makeatother

\tolerance 1414
\hbadness 1414
\emergencystretch 1.5em
\hfuzz 0.3pt
\widowpenalty=10000
\displaywidowpenalty=10000
\clubpenalty=5000
\interfootnotelinepenalty=9999
\brokenpenalty=2000
\vfuzz \hfuzz
%%%\raggedbottom


% Autorkennung:   Wie soll's aussehen?
\providecommand{\autor}[1]{\hfill\textbf{#1}}

\endinput
