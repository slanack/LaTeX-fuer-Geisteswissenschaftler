% !TeX root = lfgw.tex
\chapter{Vorwort}

Der aus dem Griechischen stammende Begriff \enquote{Philologie} bedeutet wörtlich übersetzt \enquote{Liebe zur Sprache}. Übeträgt man dies auf die Philologen selbst, so müssten Philologen besonder viel Lieber und Sorgfalt bei der Gestaltung ihrer Texte investieren. So zumindest die Theorie, die Wirklichkeit sieht leider anders aus. Auch in den Geisteswissenschaften setzt sich das Schreibprogramm eines großen amerikanischen Konzerns durch, obwohl gerade Microsoft Word für viele philogische Text-Anforderungen in vielen Fällen nicht die optimale Wahl ist, da benötigte Funktionen entweder überhaupt nicht oder nur in unnötig komplizierter Form vorhanden sind. 

Dieses Skript richtet sich an zwei Zielgruppen, deren Bedürfnisse verwandt, aber doch verschieden sind:

\begin{itemize}
 \item Angehörige der Geistes- und Sozialwissenschaften, die \LaTeX{} im Rahmen ihrer Arbeit
 -- sei es zur Erstellung einer Seminararbeit, einer Bachelor- oder Masterarbeit, eines
 Dissertationsprojektes oder eines komplexen Editionsvorhabens --
 einsetzen möchten und keine oder nur sehr geringe Kenntnisse des \LaTeX{}-Systems haben.
 \item Erfahrene \TeX-niker
\end{itemize}

Am Beginn dieses Buches muss ich den Leser um Vergebung für eine eigentlich unverzeihliche begriffliche
Unsauberkeit bitten: nichtmathematische Geisteswissenschaftler!

\LaTeX{} wird von einer sehr aktiven Community weiterentwickelt.

DTK verzeichnet ca. 50 NEUE Pakete im Quartal.

Deshalb große Unübersichtlichkeit: Neulinge finden oft auch veraltetes bzw,. sehen vor lauter
Wald die Bäume nicht.

Deshalb Idee dieses Buches: Für die häufigsten Bedürfnisse im geisteswissenschaftlichen
Bereich gangbare, aktuelle Lösungen vorstellen.

Nähere Infos zur Intensiveren Nutzung enthalten immer die Paketdokumentationen.

Herzlichen Dank 
\endinput