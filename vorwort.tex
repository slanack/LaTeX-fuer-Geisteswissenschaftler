% !TeX root = lfgw.tex
\chapter{Vorwort}

Philologie bedeutet

also müssten Philologen besondere Sorgfalt und liebe zur Erstellung ...

Dieses Skript richtet sich an zwei Zielgruppen, deren Bedürfnisse verwandt, aber doch verschieden sind:

\begin{itemize}
 \item Angehörige der Geistes- und Sozialwissenschaften, die \LaTeX{} im Rahmen ihrer Arbeit
 -- sei es zur Erstellung einer Seminararbeit, einer Bachelor- oder Masterarbeit, eines
 Dissertationsprojektes oder eines komplexen Editionsvorhabens --
 einsetzen möchten und keine oder nur sehr geringe Kenntnisse des \LaTeX{}-Systems haben.
 \item Erfahrene \TeX-niker
\end{itemize}

Am Beginn dieses Buches muss ich den Leser um Vergebung für eine eigentlich unverzeihliche begriffliche
Unsauberkeit bitten: nichtmathematische Geisteswissenschaftler!

\LaTeX{} wird von einer sehr aktiven Community weiterentwickelt.

DTK verzeichnet ca. 50 NEUE Pakete im Quartal.

Deshalb große Unübersichtlichkeit: Neulinge finden oft auch veraltetes bzw,. sehen vor lauter
Wald die Bäume nicht.

Deshalb Idee dieses Buches: Für die häufigsten Bedürfnisse im geisteswissenschaftlichen
Bereich gangbare, aktuelle Lösungen vorstellen.

Nähere Infos zur Intensiveren Nutzung enthalten immer die Paketdokumentationen.

Herzlichen Dank 
\endinput