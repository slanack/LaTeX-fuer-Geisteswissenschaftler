% !TEX root = lfgw.tex
\chapter{Von WYSIWYG zu \LaTeX}
\autor{Axel Kielhorn}

\noindent Am Anfang war \Program{Word}\footnote{Frei nach Johannes 1:1}.
Obwohl es eine große Auswahl an Textverarbeitungsprogrammen gibt
(\Program{Word Perfect}\footnote{Ja, das gibt es noch.}, \Program{LibreOffice Writer}, \Program{Textmaker}, \Program{Pages}, \dots),
ist für viele Textverarbeitung ein Synonym für \Program{MS Word}%
\footnote{Im Süden der Republik gab es bis vor kurzem noch ein kleines Dorf 
das sich gegen die Omnipräsenz von MicroSoft gewehrt hat,
aber auch diese Aufständischen haben inzwischen aufgegeben.}.

Für den Anfang reicht das sicherlich auch aus, aber irgendwann kommt der Punkt an dem die Anforderungen die Fähigkeiten von \Program{MS Word} überschreiten.
Wie bekomme ich die paar Dutzend Seiten nun ohne viel Aufwand nach \LaTeX{{}?

\section{Hilfe ist nah!}
\dictum[Initiative Safer Text]{Ich mach's mit \LaTeX}

Damit eine automatische Konvertierung funktioniert, muss das Originaldokument eine Struktur enthalten:

\begin{itemize}
\item Überschriften müssen als Überschriften formatiert sein, nur groß und fett reicht nicht.
\item Überschriften sollten automatisch nummeriert werden.
\item Aufzählungen sollten als solche markiert sein, \Program{Word} macht das inzwischen automatisch.
\end{itemize}

\subsection{Der interaktive Weg}

Interaktiv ist ein modernes Wort für manuell, d.\,h. ich mache die ganze Arbeit selbst. 
Im letzten Jahrhundert gab es dazu keine Alternative, daher als erstes ein kurzer Rückblick:

\begin{itemize}
\item Gesamten Text im Quellprogramm markieren und kopieren.
\item Text im Editor der Wahl einfügen.
\item \LaTeX{}-Befehle ergänzen.
\item Fertig!
\end{itemize}

Alternativ kann man den Text auch als reinen Text speichern, wenn möglich sollte man hier als Kodierung UTF-8 verwenden.

\subsection{Writer2LaTeX}

Es gibt eine Erweiterung für \Program{LibreOffice} mit der man ein Dokument nach \LaTeX{} exportieren kann.
Diese kann man unter \url{http://writer2latex.sourceforge.net} herunterladen.
Zusätzlich wird noch \Program{Java} benötigt.
Unter \texttt{Exportieren \dots} kann man jetzt das Format \LaTeX{} auswählen.

Beim Export sollte man die Einstellung \enquote{Sehr aufgeräumter Artikel} wählen, als Encoding \enquote{Unicode (UTF8)}.
Das Ergebnis ist ein \LaTeX{}-Dokument in dem Aufzählungen und Tabellen erhalten bleiben, jegliche Formatierung jedoch verloren geht.

Normalerweise werden Bilder aus dem Dokument automatisch extrahiert.
Funktioniert dies nicht, kann es helfen, die Option \enquote{Float Abbildungen} auszuschalten.

\subsection{Pandoc}

Pandoc \url{http://pandoc.org} ist ein universeller Dokumentenkonvertierer, sozusagen der GraphicConverter für Texte.
Das Programm kann \texttt{.docx}, \texttt{.odt} und \texttt{.tex} Dateien lesen und schreiben.
Hier darf man aber keine Wunder erwarten, bei \LaTeX{}-Dateien werden nur relativ einfache Dokumente erfolgreich verarbeitet.

\begin{lfgwcode}{label={lst:pandoc-befehlszeile}}
pandoc Dokument.docx --extract-media=. -s -o Dokument.tex
\end{lfgwcode}

Die Option \texttt{-{}-extract-media=.} wählt das aktuelle Verzeichnis \enquote{.} als Ablageort für Bilddateien.
Ohne diese Option werden keine Bilder extrahiert.
Die Dateien werden in einen Ordner \texttt{media} verpackt.
Die Option \texttt{-s} erzeugt ein \enquote{standalone} Dokument, also eine komplette \LaTeX{}-Datei die direkt
mit \Program{pdflatex} oder \Program{lualatex} übersetzt werden kann.
Wenn im Dokument nicht-lateinische Buchstaben verwendet werden, muss noch ein passender Zeichensatz gewählt werden.
Der Standard-Zeichensatz Latin Modern enthält nur lateinische Buchstaben.

\Program{Pandoc} ließt keine Dateien im alten \texttt{.doc} Format.

Die Option \texttt{.docx} und \texttt{.odt} zu lesen ist relativ neu.
In diversen Linux-Distributionen befinden sich noch alte Versionen ohne diese Fähigkeit.
Beim Erscheinen dieses Buches war die Version 1.19 aktuell und 2.0 in der Entwicklung.

Auch Pandoc hat manchmal Probleme mit eingebundenen Bildern.
Da man beim eigenen Werk in der Regel die Originalbilder zur Verfügung hat, ist dies nicht unbedingt ein Problem.

\subsection{\TeX{}-Stammtisch}

In vielen Städten gibt es einen \TeX{}-Stammtisch mit mehr oder weniger erfahrenen \LaTeX{}-Nutzern.
Evtl. trifft man hier jemanden, der bereit ist für ein Bier oder ein Abendessen bei der Konvertierung zu helfen.
Außerdem hat man so einen Experten vor Ort.

\subsection{Empfehlung}

Beide Programme liefern gute Ergebnisse. 
Die Dokumentstruktur wird übernommen, Fußnoten werden erkannt und als \cs{footnote}-Befehle eingefügt.
Diese kann man dann in die entsprechenden \paket{reledmac} Fußnoten umbennenen.

Wer keine Probleme mit der Kommandozeile hat, 
erhält mit \Program{pandoc} ein brauchbares Werkzeug und erspart sich den Download von einigen hundert MB. 
Wer bereits \Program{LibreOffice} installiert hat, kann mit dem Plugin \Program{Writer2 Latex} die Texte exportieren
und spart sich den Weg zur Kommandozeile.

\section{Nacharbeit}

Die konvertierten Texte sind eine Rohfassung.
Auch wenn das Dokument fehlerfrei übersetzt wird, sind noch ein paar Anpassungen erforderlich:

\begin{itemize}
\item Die verwendete Schrift muss angepasst werden. 
Normalerweise verwendet \LaTeX{} die Schrift Latin Modern.
Diese Schrift enthält keine griechischen oder kyrillischen Zeichen.
Das Kapitel \ref{sec:normale-schriften} enthält eine Übersicht über freie Schriften und deren Zeichenumfang.

\item Bilder müssen mit einer \cs{caption} versehen und evtl. skaliert werden.
Siehe Kapitel Bilder\marginpar{Kapitel Bilder?}
\end{itemize}

