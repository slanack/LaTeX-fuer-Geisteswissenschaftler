% !TeX root = lfgw.tex

\chapter{Kurzbiografien der Autoren}

\paragraph{Lukas C. Bossert} (*1985) studierte in Konstanz und Berlin (Humboldt-Universität) Kulturwissenschaften der Antike und Klassische Archäologie. Seit 2016 in der Redaktion der Zentrale des Deutschen Archäologischen Instituts, wo er für den Bereich \emph{digitales Publizieren} zuständig ist.
Er fand mit Beginn der Doktorarbeit seine Begeisterung für \LaTeX{} und treibt mit \emph{digitales-altertum|de} die Verbreitung in den  Altertumswissenschaften voran, mit speziellen Paketen, Workshops oder der \TeX nischen redaktionellen Betreuung von Zeitschriften in der Archäologie.

\paragraph{Axel Kielhorn} hat Maschinenbau studiert und dabei \LaTeX{} kennengelernt. 
Seit 2011 schreibt er vorwiegend in Markdown und konvertiert dieses mit \Program{pandoc} nach \LaTeX{} oder ePub.

\paragraph{Thomas Hilarius Meyer} (* 1980) hat in Tübingen katholische Theologie, Geschichte und Germanistik studiert und arbeitet seit 2005 als Lehrer an Gymnasien und Gemeinschaftsschulen im Saarland und in Mecklenburg-Vorpommern; er ist verheiratet, hat sechs Kinder und züchtet Bienen.
Aus purem Spaß spielt er in seiner Freizeit gerne mit Perl, \LaTeX{} und einem riesigen Stapel von Sekundärliteratur zu Goethes Faust. Kein Wunder, dass seine Promotion über die Integration magischer Vorstellungen in das wissenschaftliche Weltbild zur Zeit der großen europäischen Hexenverfolgung so schnell nicht fertig wird\ldots

\paragraph{Craig Parker-Feldmann} Lorem ipsum dolor sit amet, consetetur sadipscing elitr, sed diam nonumy eirmod tempor invidunt ut labore et dolore magna aliquyam erat, sed diam voluptua. At vero eos et accusam et justo duo dolores et ea rebum. Stet clita kasd gubergren, no sea takimata sanctus est Lorem ipsum dolor sit amet

\paragraph{Philipp Pilhofer} (* 1986) hat evangelische Theologie studiert und fast zehn Jahre für eine tschechisch-australische Softwarefirma gearbeitet. Seit 2017 ist er wissenschaftlicher Mitarbeiter am Lehrstuhl für Ältere Kirchengeschichte an der Humboldt-Universität zu Berlin. Er hat \TeX{}nische Erfahrungen bei neun gedruckten Büchern mit insgesamt mehreren Tausend Druckseiten gesammelt.

\paragraph{Dr. Christine Römer} Lorem ipsum dolor sit amet, consetetur sadipscing elitr, sed diam nonumy eirmod tempor invidunt ut labore et dolore magna aliquyam erat, sed diam voluptua. At vero eos et accusam et justo duo dolores et ea rebum. Stet clita kasd gubergren, no sea takimata sanctus est Lorem ipsum dolor sit amet

\paragraph{Martin Sievers} (* 1977) studierte Angewandte Mathematik in Trier und arbeitete zunächst als selbständiger Dienstleister im Bereich \enquote{Wissenschaftlicher Textsatz}. Seit 2010 ist er wissenschaftlicher Mitarbeiter am \enquote{Trier Center for Digital Humanities} und betreut dort u.\,a. den \TeX-Export der virtuellen Forschungsumgebung FuD sowie weitere \TeX-gestützte Publikationen.

\paragraph{Dr. Uwe Ziegenhagen} (* 1977) stammt aus dem Berliner Umland und hat Betriebswirtschaftslehre und Statistik studiert. Nach der Promotion hat es ihn nach Köln verschlagen, wo er seitdem in verschiedenen Finanzunternehmen gearbeitet hat.