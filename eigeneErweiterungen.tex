% !TeX root = lfgw.tex
\chapter{Eigene \LaTeX-Erfindungen dokumentieren}%%%MS: Erfindungen finde ich den falschen Begriff

Man muss kein besonders gewiefter \TeX{}niker sein, um in die Verlegenheit zu kommen, eigene 
\LaTeX-\enquote{Erfindungen} dokumentieren zu müssen
-- und sei es nur ein einzelnes kleines \lstinline/\renewcommand/
im Rahmen eines Projektes mit mehreren Mitarbeitern.

Auf den zweiten Blick ist dies für Geisteswissenschaftler gar nicht so anders, 
als die anderen Dinge auch, die Philologen mit Textverarbeitungsprogrammen so anstellen:

dokumentierter text  ---  dokumentierender text ...


\minisec{Wiedergabe der \TeX-typischen Logos}

Das Paket \paket{hologo} von Heiko Oberdiek stellt den Befehl \lstinline/\hologo{Name}/ zur 
Verfügung, das u.\,a. folgende Logos erzeugen kann:

\begin{center}
 \begin{tabular}{ll}
  (La)TeX & \hologo{(La)TeX} \\
  AmSLaTeX & \hologo{AmSLaTeX} \\
  AmSTeX & \hologo{AmSTeX} \\
  biber & \hologo{biber} \\
  BibTeX & \hologo{BibTeX} \\
  HanTheThanh & \hologo{HanTheThanh} \\
  KOMAScript & \hologo{KOMAScript} \\
  La & \hologo{La} \\
  LaTeX & \hologo{LaTeX} \\
  LaTeX2e & \hologo{LaTeX2e} \\
  LaTeX3 & \hologo{LaTeX3} \\
  LaTeXe & \hologo{LaTeXe} \\
  LuaLaTeX & \hologo{LuaLaTeX} \\
  LuaTeX & \hologo{LuaTeX} \\
  LyX & \hologo{LyX} \\
  METAFONT & \hologo{METAFONT} \\
  MetaFun & \hologo{MetaFun} \\
  METAPOST & \hologo{METAPOST} \\
  MetaPost & \hologo{MetaPost} \\
  MiKTeX & \hologo{MiKTeX} \\
  teTeX & \hologo{teTeX} \\
  TeX & \hologo{TeX} \\
  Xe & \hologo{Xe} \\
  XeLaTeX & \hologo{XeLaTeX} \\
  XeTeX & \hologo{XeTeX} \\
 \end{tabular}

\end{center}





\minisec{Listings einbinden}

\paket{listings}


\minisec{Latex-Quelltext und seine Ausgabe wiedergeben}

\paket{showexpl}
