% !TeX root = lfgw.tex
\chapter{Literatur und Zitate automatisch verwalten}
\dictum[U. Eco, Name der Rose]{...}
\label{biblatex}
\autor{Lukas C. Bossert, Axel Kielhorn}

\section{Der (neue) Standard: \paket{biblatex} und \paket{biber}}
Während in den naturwissenschaftlichen Bereichen die Bibliographie eine
weniger relevante Rolle spielt -- als beispielsweise in den Geisteswissenschaften --
hält sich dort die (veraltete) Technik der Literaturverwaltung mittels \BibTeX.

In den letzten Jahren hat sich jedoch allmählich und unumkehrbar die Möglichkeit durchgesetzt Bibliographien und Zitate mittels dem Paket \paket{biblatex} zu erstellen.\footcite{wassenhoven:dtk2008/2,wassenhoven:dtk2008/4}
Der Vorteil ist die einfache(re) Programmierung, die leichte Einbindung und der Wechsel verschiedener Bibliographie- und Zitierstile.

Das folgende Kapitel zeigt die Arbeit mit \paket{biblatex}: 
Zunächst wird erläutert, wie man eine Bibliografie-Datenbank erstellt und verwaltet (\cref{sec:bibliografiedatenbank}),
anschließend wie man die notwendigen Felder bestimmter Eintragstypen (Buch, Zeitschrift, Lexikon etc.) belegt, 
schließlich wie man im eigenen Text die Verweise einbaut (\cref{sec:zitate}) und
zum Schluss wie man eine oder mehrere (Teil"~)""Bibliografien erstellt.

Zuletzt werden verschiedene Bibliografie-Stile vorgestellt,
die besonders für Geisteswissenschaftler relevant sind und die auf unterschiedliche Anforderungen in den verschiedenen Fachrichtungen Lösungen bereithalten (\cref{sec:bibliografie}).


\paket{biblatex} wurde 2006 %JAHR  einfügen
von Philipp Lehmann entworfen und seitdem stetig weiterentwickelt. 
Von Haus aus bringt es bereits eine Großzahl an Einstellungen mit, 
die es erlauben auf die Vorgaben in den einzelnen Disziplinen einzugehen. 
Für weitere spezielle Eigenheiten in der Zitation und Bibliographie haben unzählige Mitglieder der \TeX -Community Stile und Erweiterungen entwickelt,
die dem  Paket als Option übergeben werden.\footcite{voss:bibliografien}


Wohingegen die offizielle Paket-Dokumentation von \paket{biblatex}  mehr als 250~Seiten umfasst und  für einen Einsteiger nur bedingt zu empfehlen ist,
gibt es ebenfalls auf englisch eine sehr ansprechende und verständliche Kurzeinführung.\footcite{biblatex-tutorial}

Eine deutsche Einführung gibt auf github\footcite{biblatex-bottcher}, 
eine deutsche Übersetzung des Handbuchs ist Bestandteil von TeX-Live und im Verzeichnis 
texmf-dist/doc/latex/translation-biblatex-de zu finden.
Die Übersetzung entspricht nicht dem aktuellen Stand, ist aber in den meisten Fällen ausreichend.
Die Änderungen kann man im englischen Original im Anhang \enquote{Revision History} verfolgen.

\section{Aufbau der Bibliografie-Datenbank}\label{sec:bibliografiedatenbank}
Eine Bibliografie-Datenbank ist zunächst nichts anderes als eine \meta{.tex}-Datei,
die allerdings die Endung \meta{.bib} hat und ebenso mit jedem Texteditor bearbeitet werden kann.

\subsection{Grundlegender Aufbau}
Am Beispiel von \cite{voss:einfuehrung} sei der Aufbau eines Eintrags in der Bibliographiedatei erklärt (\cref{{lis:voss:einfuehrung}}):

\begin{lfgwcode}{label={lis:voss:einfuehrung}}
@Book{voss:einfuehrung,
 author    = {Herbert Voß}, 
 title     = {Einführung in \LaTeX},
 publisher = {DANTE~e.V. and Lehmanns Media},
 location  = {Berlin and Heidelberg},
 year      = {2016},
 edition   = {2},
}
\end{lfgwcode}

\begin{description}
 \item[1] \meta{@Book}: Damit wird das Wesen des Werks, der Publikationstypus, definiert, in diesem Fall handelt es sich um ein Buch; vgl. \cref{lit:publikationstypus}.
 \item[1] \meta{voss:einfuehrung}: Dies ist der \meta{Schlüssel}, den jeder Eintrag haben muss, um im Textdokument zitiert werden zu können.
 \item[2-7] Alle Informationen zu einem Eintrag müssen in bestimmten Feldern geschrieben werden; z.\,B. \meta{author = {Herbert Voß}}: 
 Der Name des Autors des Buches wird in das Feld \marg{author} geschrieben. 
\end{description}
Jeder Eintrag beginnt mit dem Publikationstypus, alle weiteren Informationen sind innerhalb eines Klammerpaares.
\subsection{Schlüsselvergabe}
Anhand des \meta{Schlüssels} wird im Textdokument über einen \cs{cite}-Befehl (\cref{lit:cite-befehle}) das Werk eingebunden.
Ein solcher \meta{Schlüssel} muss innerhalb einer Bibliografie-Datei einmalig und eindeutig sein, 
da es ansonsten zu einer Fehlermeldung kommt.
In der Regel bieten Bibliografieverwaltungsprogramme wie JabRef eine Funktion an, anhand der man den \meta{Schlüssel} erzeugen kann. 

Es bietet sich an, den Namen des Autors und ein Kurztitel oder Nachname des (ersten) Autors und das Jahr der Publikation. 
Je kürzer und prägnanter ein \meta{Schlüssel} ist, desto weniger Tipparbeit bedeutet dies im Textdokument, zumal man den \meta{Schlüssel} zwar jederzeit ändern kann, aber diese Änderungen ebenfalls im Textdokument ausführen muss. 
Normalerweise bleibt daher der \meta{Schlüssel} für das gesamte Dokument gleich.

\subsection{Publikationstypen und ihre Datenfelder}\label{lit:publikationstypus}
Standardmäßig unterscheidet das Paket \Paket{biblatex} 29 %Anzahl
Publikationstypen, die je nach Typ unterschiedlich ausgegeben werden.
Zusätzliche Bibliografiestile können weitere Publikationstypen bereithalten,
wie z.\,B. der Stil \meta{arthistory-bonn} für einen ›Ausstellungskatalog‹ den Eintrag \meta{@exhibcatalog} und \meta{@movie} für einen Film im Stil \meta{fiwi}.

Zu jedem Publikationstypus gibt es notwendige und optionale Felder.
Grundsätzlich ist es eine gute Idee alle bekannten Informationen an einer Stelle (der Literaturdatenbank) zu sammeln,
auch wenn der verwendete Stil diese Information nicht nutzt.

Die meisten Publikationstypen benötigen folgende Felder:
\meta{author} oder \meta{editor}
\meta{title},
\meta{year} oder \meta{date}.

\subsubsection{Bücher (Monografie, Sammmelband)}

Zu den buchartigen Werken gehören: 

\begin{labeling}{\meta{@mvproceedings}}
\item[\meta{@book}]         Ein einzelnes Buch mit einem oder mehreren Autoren die für das Gesamtwerk verantwortlich sind.
\item[\meta{@mvbook}]       Ein mehrbändiges Buch.
%
\item[\meta{@collection}]   Ein Buch mit mehren Autoren, die für einzelne Abschnitte verantwortlich sind.
        Die einzelnen Abschnitte werden über \meta{@incollection} identifiziert. Das Geamtwerk hat einen \meta{editor}.
\item[\meta{@mvcollection}] Eine mehrbändige \meta{@collection}.
%
\item[\meta{@reference}]    Eine Möglichkeit Lexika hervorzuheben, 
        wird jedoch in den Standardstilen wie eine \meta{collection} behandelt.
\item[\meta{@mvreference}]
%
\item[\meta{@proceedings}]  Ein Tagungsband. Anders als bei einer \meta{collection} ist hier kein \meta{editor} erforderlich.
        Optional kann eine \meta{organization} angegeben werden.
\item[\meta{@mvproceedings}] Ein mehrbändiger Tagungsband.
%
\item[\meta{@inbook}]       Ein Teil eines Buches.
\item[\meta{@incollection}] Ein Teil einer \meta{@collection}.
\item[\meta{@inreference}]
\item[\meta{@proceedings}]
%
\item[\meta{@booklet}]      Ein Buch, das nicht von einem Verlag verlegt wurde.
        Anstelle eines \meta{author}s kann auch ein \meta{editor} angegeben werden.
        Die Art der Veröffentlichung kann im Feld  \meta{howpublished} angegeben werden
\item[\meta{@manual}]       Ein Handbuch. Anders als beim \meta{@booklet} können \meta{author} und \meta{editor} entfallen.
\item[\meta{@misc}]         Bei diesem Typus können \meta{author}, \meta{editor} und \meta{year} entfallen.
\item[\meta{@patent}]       Die Patentnummer \meta{number} ist zusätzlich erforderlich.
\item[\meta{@online}]       Anstelle eines \meta{author}s kann auch ein \meta{editor} angegeben werden.
        Zusätzlich ist die Angabe einer \meta{url} erforderlich.
\item[\meta{@report}]       Zusätzlich ist die Angabe einer \meta{institution} und des \meta{type} erforderlich.
\item[\meta{@thesis}]       Eine Dissertation, die erforderlichen Daten entsprechen dem \meta{report}.
\item[\meta{@unpublished}]  Ein noch nicht veröffentlichtes Buch.
\end{labeling}

\subsubsection{Zeitschriften (Artikel, Rezensionen)}

\begin{labeling}{\meta{@mvproceedings}}
\item[\meta{@article}]    Ein Artikel, der in einer Zeitschrift erschienen ist.
        Die Zeitschrift wird im Feld \meta{journaltitle} angegeben.
\item[\meta{@review}]     Eine Rezension (von Büchern, Filmen etc.)
        Kann von bestimmten Stilen besonders formatiert werden, 
        ist in den Standardstilen identisch mit \meta{article}
\item[\meta{@periodical}] Die komplette Zeitschrift, der \meta{editor} ist optional.
\end{labeling}

\subsubsection{Gesetze und Normen}

Für die folgenden Typen gibt es in den Standardstilen keine besondere Formatierung,
sie werden als \meta{misc} behandelt.
Spezielle fachbezogenen Bibliografiestile können eine angepasste Formatierung benutzen.

\begin{labeling}{\meta{@mvproceedings}}
\item[\meta{@legislation}]      Gesetzestexte
\item[\meta{@commentary}]       Kommentare zu Gesetzestexten
\item[\meta{@standard}]         Normen
\end{labeling}

\subsubsection{Übersicht: Datenfelder}
Theoretisch und technisch können (fast) alle Datenfelder für alle Publikationstypen genutzt werden.
Allerdings ist dies nicht immer sinnvoll, wie z.\,B. eine Seitenangabe bei einer Webseite keinen logischen Mehrwert hat.

Neben den hier aufgeführten Feldern gibt es noch weitere, die ein Dokument
genauer beschreiben können.

\begin{labeling}{\meta{organization}}
\item[Namen]    \paket{biblatex} betreibt einen großen Aufwand um Namen richtig zu formatieren.
        Ein Name besteht aus dem Vornamen \meta{first} und dem Nachname \meta{last}.
        Zusätzlich kann er noch einen Namenszusatz \meta{von} und einen Namensanhang \meta{Jr} erhalten.
        Namen werden für \meta{author} und \meta{editor} verwendet.
        Mehrere Namen werden durch ein \meta{and} verbunden.
\item[Datum] Ein Datum kann gemäß ISO 8601 angegeben werden (yyyy-mm-dd). 
\end{labeling}

\begin{labeling}{\meta{organization}}
\item[\meta{title}]       Der Titel des Werks, in den meisten Fällen erforderlich.
\item[\meta{author}]      Der Autor des Werks, in den meisten Fällen erforderlich.
\item[\meta{editor}]      Der Herausgeber, wird in einigen Typen anstelle des \meta{author}s benutzt.
\item[\meta{translator}]  Der Übersetzer.
\item[\meta{publisher}]   Der Verlag.
\item[\meta{organization}] Alternativ zu \meta{publisher} z.\,B. eine Organisation oder eine Firma.
\item[\meta{institution}] Alternativ zu \meta{publisher} z.\,B. eine Schule oder Universität.
\item[\meta{type}]        Art des \meta{report}s oder der \meta{thesis}.
%
\item[\meta{chapter}]     Das zitierte Kapitel.
\item[\meta{pages}]       Eine Seitenzahl oder ein Bereich.
%
\item[\meta{year}]        Erscheinungsjahr (bei Büchern).
\item[\meta{month}]       Erscheinungsmonat (bei Zeitschriften).
\item[\meta{issue}]       Die Ausgabe eines Buches oder einer Zeitschrift.
        Kann eine Zahl oder eine Text sein.
\item[\meta{number}]      Die Nummer eines Buches oder einer Zeitschrift in einer Serie.
        Kann eine Zahl oder eine Text sein.
%
\item[\meta{isbn}]  International Standard Book Number. 
\item[\meta{issn}]  International Standard Serial Number. 
\end{labeling}

\section{Zitate}\label{sec:zitate}\label{lit:cite-befehle}

Normale Zitate werden mit dem Befehl \cs{cite} ausgeführt:
\begin{lfgwcode}{label={lis:XXX}}
\cite*@\oarg{Präfix}\oarg{Suffix}\marg{Schlüssel}@*
\end{lfgwcode}

Während \meta{Präfix}  eine kurze Anmerkung \emph{vor} die Zitation (z.\,B. \enquote{Vgl.}) setzt, 
wird  \meta{Suffix} für gewöhnlich für Seitenzahlen verwendet.
Ist nur ein optionales Argument definiert, 
dann wird es als \oarg{Suffix} behandelt.
\begin{lfgwcode}{label={lis:code:cite}}
\cite*@\oarg{Suffix}\marg{Schlüssel}@*
\end{lfgwcode}
Der \meta{Schlüssel} korrespondiert mit dem Schlüssel der Bibliografie-Datei.

\begin{lfgwexample}{label={lis:example:cite}}
\enquote{Der öffentliche Raum ist Teil einer Stadt.}\cite{Osland2016}.
\end{lfgwexample}

\minisec{\cs{cites}}
Möchte man hingegen mehrere Autoren oder Werke zitieren, 
gibt es zwei Möglichkeiten:
Entweder kann man dies durch die Komma-getrennte Reihung der \marg{Schlüssel} machen,
was jedoch den Nachteil hat, dass man für die einzelnen \marg{Schlüssel} keine \oarg{Präfixe} oder \oarg{Suffixe} definieren kann.

Die andere Möglichkeit sieht vor, den Befehl \cs{cites} zu verwenden, 
bei dem für jeden Autor\,/\,jedes Werk sowohl \oarg{Präfixe} als \oarg{Suffixe} definiert werden kann.
Zudem lässt sich für die gesamte Reihung ein \oarg{Präfix} und ein \oarg{Suffix} festlegen:
\begin{lfgwcode}{label={lis:code:cites}}
\cites(Prä-Präfix)(Suf-Suffix)
  *@\oarg{Präfix}\oarg{Suffix}\marg{Schlüssel}@*%
  *@\oarg{Präfix}\oarg{Suffix}\marg{Schlüssel}@*%
  *@\oarg{Präfix}\oarg{Suffix}\marg{Schlüssel}\ldots@*
\end{lfgwcode}
\begin{lfgwexample}{label={lis:example:cites}}
Der öffentliche Raum ist Teil einer Stadt \cites(vgl.)(){Osland2016} {Evangelidis2014}.
\end{lfgwexample}
% Leerzeichen vor Evangelidis2014 erforderlich, sonst kein Umbruch (Axel).
 
 
\minisec{\cs{parencite}}
Manchmal soll die Zitation in Klammern stehen.
Um dies nicht händisch in runde Klammern setzen zu müssen (und ggf. die \enquote{Klammerschachtelregel} zu verletzen),
kann dafür der Befehl  \cs{parencite} verwendet werden:
\begin{lfgwcode}{label={lis:code:parencite}}
\parencite*@\oarg{Suffix}\marg{Schlüssel}@*
\end{lfgwcode} 
Mit diesem Zitationsbefehl wird die korrekte Ordnung von korrespondierenden Klammern berücksichtigt.
\begin{lfgwexample}{label={lis:example:parencite}}
\enquote{Der öffentliche Raum ist Teil einer Stadt.} \parencite{Osland2016}
\end{lfgwexample}

\minisec{\cs{parencites}}
Ebenso lassen sich auch mehrere Zitationen mit Klammern umschließen.
Dies wird mittels \cs{parencites} umgesetzt:
\begin{lfgwcode}{label={lis:code:parencites}}
\parencites(Prä-Präfix)(Suf-Suffix)%
*@\oarg{Präfix}\oarg{Suffix}\marg{Schlüssel}@*%
*@\oarg{Präfix}\oarg{Suffix}\marg{Schlüssel}@*%
*@\oarg{Präfix}\oarg{Suffix}\marg{Schlüssel}\ldots@*
\end{lfgwcode}
\begin{lfgwexample}{label={lis:example:parencites}}
Der öffentliche Raum ist Teil einer Stadt.\parencites(s.)(){Osland2016}
[vgl.][]{Evangelidis2014}.
\end{lfgwexample}
% "%" vor Evangelidis2014 nicht erforderlich, irritiert im Beispiel (Axel).

\minisec{\cs{textcite}}
Neben den bereits angeführten \cs{cite}-Befehlen gibt es eine dritte Möglichkeit der Zitationsangabe:
\cs{textcite} ist vor allem für die Fälle zu nutzen, 
bei denen der Autor\,/\,das Werk im Fließtext genannt sein soll, 
aber die weiteren Angaben (Publikationsjahr, Seitenangabe) nur in runden Klammern dahinter.
\begin{lfgwcode}{label={lis:code:textcite}}
\textcite*@\oarg{Suffix}\marg{Schlüssel}@*
\end{lfgwcode} 

\begin{lfgwexample}{label={lis:example:textcite}}
Der öffentliche Raum ist Teil einer Stadt, sagt \textcite{Osland2016}.
\end{lfgwexample}

\minisec{\cs{textcites}}
Wiederum können mehrere Autoren\,/\,Werke mittels \cs{textcites} gelistet werden:
\begin{lfgwcode}{label={lis:code:textcites}}
\textcites(Prä-Präfix)(Suf-Suffix)%
  *@\oarg{Präfix}\oarg{Suffix}\marg{Schlüssel}@*%
  *@\oarg{Präfix}\oarg{Suffix}\marg{Schlüssel}@*%
  *@\oarg{Präfix}\oarg{Suffix}\marg{Schlüssel}\ldots@*
\end{lfgwcode}
\begin{lfgwexample}{label={lis:example:textcites}}
Der öffentliche Raum ist Teil einer Stadt, sagen \textcites{Osland2016}
[vgl.][]{Evangelidis2014}.
\end{lfgwexample}
% "%" vor Evangelidis2014 nicht erforderlich, irritiert im Beispiel (Axel).

\minisec{\cs{footcite}}
Darüberhinaus gibt es weitere \cs{cite}-Befehle, 
die die Einbettung der Zitation beeinflussen. 
Zunächst kann man mit \cs{footcite} die Zitation direkt als eigene Fußnote setzen:
 \begin{lfgwcode}{label={lis:code:footcite}}
\footcite*@\oarg{Präfix}\oarg{Suffix}\marg{Schlüssel}@*
\end{lfgwcode}
\begin{lfgwexample}{label={lis:example:footcite}}
\enquote{Der öffentliche Raum ist Teil einer Stadt.}\footcite{Osland2016}
\end{lfgwexample}
\cs{footcite} ist das Äquivalent zu \lstinline/\footnote{\cite{Osland2016}.}/
was jedoch manche (überflüssige) Tipparbeit spart.

\minisec{\cs{footcites} }
Für mehrere Autoren\,/\,Werke in einer Fußnote gibt es auch \cs{footcites}:
\begin{lfgwexample}{label={lis:example:footcites}}
\enquote{Der öffentliche Raum ist Teil einer Stadt.}\footcites(s.)(){Osland2016}
[vgl.][]{Evangelidis2014}
\end{lfgwexample}
% "%" vor Evangelidis2014 nicht erforderlich, irritiert im Beispiel (Axel).
 
\minisec{\cs{smartcite}}
Eine clevere Art und Weise Zitationen als Fußnote zu setzen,
bietet der Befehl \cs{smartcite}.
 \cs{smartcite} reagiert auf die Umgebung des Befehls:
 Befindet sich \cs{smartcite} innerhalb des Fließtexts wird es wie \cs{footcite} behandelt 
 und die Zitation in eine Fußnote setzen. 
Wird die Zitation allerdings in einer Fußnote aufgerufen,
erfolgt die Ausgabe nach dem Schema von \cs{cite}. 
\begin{lfgwcode}{label={lis:code:smartcite}}
\smartcite*@\oarg{Suffix}\marg{Schlüssel}%@*
\end{lfgwcode} 

\begin{lfgwexample}{label={lis:example:smartcite}}
Der öffentliche Raum ist Teil einer Stadt.\smartcite{Osland2016} 
Eventuell aber zugangsbeschränkt. \footnote{\smartcite[vgl.][] {Evangelidis2014}.}
\end{lfgwexample}
%%% Smartcite in der Fußnote funktioniert hier nicht. 

\minisec{\cs{smartcites}}
Wiederum gibt es analog auch den Befehl \cs{textcites}, 
um mehrere Autoren\,/\,Werke clever zu zitieren:
\begin{lfgwcode}{label={lis:code:smartcites}}
\smartcites(Prä-Präfix)(Suf-Suffix)%
  *@\oarg{Präfix}\oarg{Suffix}\marg{Schlüssel}@*%
  *@\oarg{Präfix}\oarg{Suffix}\marg{Schlüssel}@*%
  *@\oarg{Präfix}\oarg{Suffix}\marg{Schlüssel}\ldots@*
\end{lfgwcode}
\begin{lfgwexample}{label={lis:example:smartcites}}
Der öffentliche Raum ist Teil einer Stadt.\smartcites{Osland2016} {Evangelidis2014} 
Eventuell aber zugangsbeschränkt.\footnote{ \smartcites{Osland2016}
[cf.][]{Evangelidis2014}.}
\end{lfgwexample}

\minisec{\cs{autocite}}
Mit  \cs{autocite} ist eine individuelle und flexible Zitationsangabe möglich,
indem man in der Präambel steuert,
wie \cs{autocite} ausgegeben werden soll.
Für gewöhnlich stehen folgende Optionen zur Verfügung:
\begin{labeling}{footnote}
	\item[plain] Ausgabe wie \cs{cite}
	\item[inline]Ausgabe wie \cs{parencite}
	\item[footnote]Ausgabe wie \cs{footcite}
\end{labeling}
\begin{lfgwcode}{label={lis:code:autocite}}
\autocite*@\oarg{Präfix}\oarg{Suffix}\marg{Schlüssel}@*
\end{lfgwcode} 

\begin{lfgwexample}{label={lis:example:autocite}}
Der öffentliche Raum ist Teil einer Stadt \autocite{Osland2016} 
\end{lfgwexample}

\minisec{\cs{fullcite} \cs{footfullcite}}
Mit den Befehlen \cs{fullcite} und \cs{footfullcite} werden zwei Möglichkeiten gegeben,
mit denen man den kompletten Bibliographie-Eintrag in den Fließtext bzw. in die Fußnote schreiben kann.
\begin{lfgwcode}{label={lis:code:footfullcite}}
\fullcite*@\oarg{Präfix}\oarg{Suffix}\marg{Schlüssel}@*
\footfullcite*@\oarg{Präfix}\oarg{Suffix}\marg{Schlüssel}@*
\end{lfgwcode} 

\begin{lfgwexample}{label={lis:example:footfullcite}}
Der öffentliche Raum ist Teil einer Stadt.\footfullcite{Osland2016}
Das steht auch geschrieben bei \fullcite{Evangelidis2014}
\end{lfgwexample}



\minisec{\cs{citeauthor} \cs{citetitle}}
Neben den \enquote{geläufigen} \cs{cite}-Befehlen kann man auch nur den oder die Autoren zitieren 
und ebenso nur den Werktitel.
Dies funktioniert für den Fließtext und für Fußnoten gleichermaßen:
\begin{lfgwcode}{label={lis:code:citeauthor}}
\citeauthor *@ \oarg{Präfix}\oarg{Suffix}\marg{Schlüssel} @*
\end{lfgwcode} 
  and for the works 
\begin{lfgwcode}{label={lis:code:citetitle}}
\citetitle *@\oarg{Präfix}\oarg{Suffix}\marg{Schlüssel} @*
\end{lfgwcode} 

\begin{lfgwexample}{label={lis:example:citetitle}}
Der öffentliche Raum ist Teil einer Stadt sagt \citeauthor{Osland2016} in \citetitle{Osland2016}.
\footnote{Der öffentliche Raum ist Teil einer Stadt sagt \citeauthor{Osland2016} in \citetitle{Osland2016}.}
\end{lfgwexample}


\minisec{\cs{nocite}}
Möchte man einen Eintrag nicht als Zitation im Text haben, 
aber auf die Auflistung in der Bibliographie nicht verzichten,
dann kann man \cs{nocite}\marg{Schlüssel} verwenden.
Die Zitation wird damit \enquote{unsichtbar} zitiert:
Sie taucht nicht im Fließtext aber dennoch in der Bibliographie.

Mit dem Befehl \cs{nocite}\marg{*} werden alle Werke aus der
Literaturdatenbank in das Literaturverzeichnis übernommen.


\section{Bibliografie-Stile}\label{sec:bibliografiestile}
Mit der zunehmenden Anwendung von \LaTeX{} in den Geisteswissenschaften steigt auch die Frage nach den \enquote*{passenden} Zitierstilen für die unterschiedlichen Fachgebiete:
Sei es Archäologie, Alte Geschichte, Filmwissenschaft, Theologie usw.
\subsection{Standardstile von biblatex}

Es wäre schön, wenn man sich auf Zitate gemäß DIN~1505, ISO~690 oder APA beschränken könnte.
Leider hat jede Zeitschrift und jedes Hochschulinstitut einen eigenen Zitierstil.
Die folgenden biblatex-Stile sollten einen Großteil der Anforderungen abdecken, oder sich leicht anpassen lassen.

\begin{labeling}{historische-zeitschrift}
  \item[biblatex-apa]       APA Style Guide (6th Edition)
  \item[biblatex-dw]        Bietet zwei Optionen: \lstinline/style=authortitle-dw/ und \lstinline/style=footnote-dw/.
  \item[biblatex-archaeology] Regeln des \enquote{Deutschen Archäologischen Instituts} (sehr viele Optionen)
  \item[geschichtsfrkl]     Regeln der Historiker der Universität Freiburg
  \item[historische-zeitschrift] Regeln der \enquote{Historischen
    Zeitschrift}
  \item[biblatex-historian] Angelehnt an das \enquote{Chicago Manual of Style}
  \item[biblatex-arthistory-bonn] Regeln des \enquote{Kunsthistorischen Instituts der Universität Bonn}.
  \item[biblatex-fiwi]      Der biblatex-Zitierstil für Filmwissenschaftler bietet zwei Optionen: \lstinline/style=fiwi/ und \lstinline/style=fiwi2/.
\end{labeling}

\minisec{Weitere nützliche Bibliografiestile}

TeX-Live enthält ca. 50 angepasste Bibliografiestile für spezielle Anwendungen, 
nicht alle davon sind für deutschsprachige Nutzer relevant.


\begin{labeling}{historische-zeitschrift}
\item[biblatex-anonymous] Zitieren anonymer Werke.
\item[biblatex-bwl]       Regeln für den BWL-Studiengang an der FU Berlin.
\item[biblatex-chem]      Für Chemiker
\item[biblatex-chicago]   Nach den Regeln des \enquote{Chicago Manual of
  Style}
\item[biblatex-ext]       Biblatex Erweiterungen
\item[biblatex-iso690]    Regeln der ISO 690
\item[biblatex-juradiss]  Speziell für Juristen
\item[biblatex-lni]       Regeln der \enquote{Lecture Notes in Informatics}
\item[biblatex-luh-ipw]   Regeln des Instituts für Politische Wissenschaft der Universität Hannover
\item[biblatex-manuscripts-philology]
\item[biblatex-mla]       Regeln der \enquote{Modern Language Association}
\item[biblatex-musuos]    Regeln des \enquote{Instituts für Musik und
  Musikwisenschaften der Universität Osnabrück}
\item[biblatex-nature]    Regeln der Zeitschrift \enquote{Nature}
\item[biblatex-oxref]     Regeln des \enquote{Oxford Guide to Style}
\item[biblatex-phys]      Regeln der AIP und der APS
\item[biblatex-sbl]       Regeln der \enquote{Society of Biblical Literature}
\item[biblatex-science]   Regeln der Zeitschrift \enquote{Science}
\item[biblatex-socialscienceshuberlin] Regeln der Sozialwissenschaften der Humboldt-Universität zu Berlin
\item[biblatex-trad]      Nachbildung der BibTeX Formatierung
\end{labeling}

\subsection{Spickzettel}

Biblatex ist inzwischen so leistungsfähig geworden, dass man sich nicht alles merken kann.
Das dachte sich auch Clea F. Rees und schuf im Jahre 2017 das \Package{biblatex-cheatsheet}.
Hier hat man alle \cs{cite} Befehle auf einen Blick und kann auch die verfügbaren Datenbank Felder schnell nachschlagen.

\section{Ein Beispiel für Historiker: Quellen und Sekundärliteratur}\label{sec:bibliografie}
\index{Quellenverzeichnis}
\index{Bibliographie}
\index{Literaturverzeichnis}
\index{Sekundärliteratur}


%\begin{lfgwcode}{}  % läuft bei mir nicht - thm; ist auch nur für die Ausgabe einer Überschrift (ms))
%\printbibheading[%
%  heading=bibliography,% Standard
%  %heading=bibnumbered,% if you want it numbered
%  title={Bibliographie}]% Überschrift für Bibliographie  
%\end{lfgwcode}


\begin{lfgwcode}{label={lis:bibantikequellen}}
\printbibliography[%
  heading=subbibliography,
  %heading=subbibnumbered,% if you want it numbered
  keyword=ancient,%
  title={Antike Quellen}]
\end{lfgwcode}

\begin{lfgwcode}{label={lis:bibabksekliteratur}}
\printbibliography[%
  heading=subbibliography,
  keyword=corpus,%
  title={Abkürzungen und Sigel}]

\printbibliography[%
  heading=subbibliography,
  notkeyword=ancient,%
  notkeyword=corpus,%
  title={Sekundärliteratur}]
\end{lfgwcode}

