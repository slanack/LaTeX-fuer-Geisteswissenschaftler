% !TeX root = lfgw.tex
\chapter{Literatur und Zitate automatisch verwalten}
\dictum[U. Eco, Name der Rose]{...}
\label{biblatex}
\autor{Lukas C. Bossert}

\section{Der (neue) Standard: \paket{biblatex} und \paket{biber}}
Während in den naturwissenschaftlichen Bereichen die Bibliographie eine weniger relevante Rolle spielt -- als bspw. in den Geisteswissenschaften -- hält sich dort die (veraltete) Technik der Literaturvewaltung mittels \BibTeX.

In den letzten Jahren hat sich jedoch allmählich und unumkehrbar die Möglichkeit durchgesetzt Bibliographien und Zitate mittels dem Paket \paket{biblatex} zu erstellen.\footcite{wassenhoven:dtk2008/2,wassenhoven:dtk2008/4}
Der Vorteil ist die einfache(re) Programmierung, die leichte Einbindung und der Wechsel verschiedener Bibliographie- und Zitierstile.

Das folgende Kapitel zeigt die Arbeit mit \paket{biblatex}: 
Zunächst wird erläutert, wie man eine Bibliografie-Datenbank erstellt und verwaltet (\cref{sec:bibliografiedatenbank}),
anschließend wie man die notwendigen Felder bestimmter Eintragstypen (Buch, Zeitschrift, Lexikon etc.) belegt, 
schließlich wie man im eigenen Text die Verweise einabut (\cref{sec:zitate}) und
zum Schluss wie man eine oder mehrere (Teil"~)""Bibliografien erstellt.

Zuletzt werden verschiedene Bibliografie-Stile vorgestellt,
die besonders für Geisteswissenschaftler relevant sind und die auf unterschiedliche Anforderungen in den verschiedenen Fachrichtungen Lösungen bereithalten (\cref{sec:bibliografie}).


\paket{biblatex} wurde 2006 %JAHR  einfügen
von Philipp Lehmann entworfen und seitdem stetig weiterentwickelt. 
Von Haus aus bringt es bereits eine Großzahl an Einstellungen mit, 
die es erlauben auf die Vorgaben in den einzelnen Disziplinen einzugehen. 
Für weitere spezielle Eigenheiten in der Zitation und Bibliographie haben unzählige Mitglieder der \TeX -Community Stile und Erweiterungen entwickelt,
die dem  Paket als Option übergeben werden.\footcite{voss:bibliografien}


Wohingegen die offizielle Paket-Dokumentation von \paket{biblatex}  mehr als 250~Seiten umfasst und  für einen Einsteiger nur bedingt zu empfehlen ist,
gibt es ebenfalls auf englisch eine sehr ansprechende und verständliche Kurzeinführung.\footcite{biblatex-tutorial}


%Vgl.\footcite[79\psqq]{rouquette:2012}




\section{Aufbau der Bibliografie-Datenbank}\label{sec:bibliografiedatenbank}
Eine Bibliografie-Datenbank ist zunächst nichts anderes als eine \meta{.tex}-Datei,
die allerdings die Endung \meta{.bib} hat und ebenso mit jedem Texteditor bearbeitet werden kann.

\subsection{Grundlegender Aufbau}
Am Beispiel von \cite{voss:einfuehrung} sei der Aufbau eines Eintrags in der Bibliographiedatei erklärt (\cref{{lis:voss:einfuehrung}}):

\begin{lfgwcode}{label={lis:voss:einfuehrung}}
@Book{voss:einfuehrung,
 author = {Herbert Voß}, 
 title = {Einführung in \LaTeX},
 publisher = {DANTE~e.V. and Lehmanns Media},
 location = {Berlin and Heidelberg},
 year = {2016},
 edition = {2},
}
\end{lfgwcode}

\begin{description}
 \item[1] \meta{@Book}: Damit wird das Wesen des Werks, der Publikationstypus, definiert, in diesem Fall handelt es sich um ein Buch; vgl. \cref{lit:publikationstypus}.
 \item[1] \meta{voss:einfuehrung}: Dies ist der \meta{Schlüssel}, den jeder Eintrag haben muss, um im Textdokument zitiert werden zu können.
 \item[2-7] Alle Informationen zu einem Eintrag müssen in bestimmten Feldern geschrieben werden; bspw. \meta{author = {Herbert Voß}}: 
 Der Name des Autors des Buches wird in das Feld \marg{author} geschrieben. 
\end{description}
Jeder Eintrag beginnt mit dem Publikationstypus, alle weiteren Informationen sind innerhalb eines Klammerpaares.
\subsection{Schlüsselvergabe}
Anhand des \meta{Schlüssels} wird im Textdokument über einen \cs{cite}-Befehl (\cref{lit:cite-befehle}) das Werk eingebunden.
Ein solcher \meta{Schlüssel} muss innerhalb einer Bibliografie-Datei einmalig und eindeutig sein, 
da es ansonsten zu einer Fehlermeldung kommt.
In der Regel bieten Bibliografieverwaltungsprogramme wie JabRef eine Funktion an, anhand der man den \meta{Schlüssel} erzeugen kann. 

Es bietet sich an, den Namen des Autors und ein Kurztitel oder Nachname des (ersten) Autors und das Jahr der Publikation. 
Je kürzer und prägnanter ein \meta{Schlüssel} ist, desto weniger Tipparbeit bedeutet dies im Textdokument, zumal man den \meta{Schlüssel} zwar jederzeit ändern kann, aber diese Änderungen ebenfalls im Textdokument ausführen muss. 
Normalerweise bleibt daher der \meta{Schlüssel} für das gesamte Dokument gleich.

\subsection{Publikationstypen und ihre Datenfelder}\label{lit:publikationstypus}
Standardmäßig unterscheidet das \Paket{biblatex} 29 %Anzahl
Publikationstypen, die je nach Typ unterschiedlich ausgegeben werden.
Zusätzliche Bibliografiestile können weitere Publikationstypen bereithalten,
wie bspw. der Stil \meta{arthistory-bonn} für einen ›Ausstellungskatalog‹ den Eintrag \meta{@exhibcatalog} und \meta{@movie} für einen Film im Stil \meta{fiwi}.


\subsubsection{Bücher (Monografie, Sammmelband)}
%@book/@inbook
%@collection/@incollection

\subsubsection{Zeitschriften (Artikel, Rezensionen)}
%@article
%@review

\subsubsection{Lexika, Handbücher}
%@reference/@inreference
\subsubsection{Internet-Ressourcen}
%@online (=@www)

\subsubsection{Übersicht: Datenfelder}
Theoretisch und technisch können (fast) alle Datenfelder für alle Publikationstypen genutzt werden.
Allerdings ist dies nicht immer sinnvoll, wie bspw. eine Seitenangabe bei einer Webseite keinen logischen Mehrwert hat.


\section{Zitate}\label{sec:zitate}\label{lit:cite-befehle}

Normale Zitate werden mit dem Befehl \cs{cite} ausgeführt:
\begin{lfgwcode}{label={lis:XXX}}
\cite*@\oarg{Präfix}\oarg{Suffix}\marg{Schlüssel}@*
\end{lfgwcode}

Während \meta{Präfix}  eine kurze Anmerkung \emph{vor} die Zitation (z.\,B. \enquote{Vgl.}) setzt, 
wird  \meta{Suffix} für gewöhnlich für Seitenzahlen verwendet.
Ist nur ein optionales Argument definiert, 
dann wird es als \oarg{Suffix} behandelt.
\begin{lfgwcode}{label={lis:code:cite}}
\cite*@\oarg{Suffix}\marg{Schlüssel}@*
\end{lfgwcode}
Der \meta{Schlüssel} korrespondiert mit dem Schlüssel der Bibliografie-Datei.

\begin{lfgwexample}{label={lis:example:cite}}
%\enquote{Beware of bugs in the above code; I have only proved it correct, not tried it.} Donald Knuth
Der öffentliche Raum ist Teil einer Stadt \cite{Osland2016}.
\end{lfgwexample}

\minisec{\cs{cites}}
Möchte man hingegen mehrere Autoren oder Werke zitieren, 
gibt es zwei Möglichkeiten:
Entweder kann man dies durch die Komma-getrennte Reihung der \marg{Schlüssel} machen,
was jedoch den Nachteil hat, dass man für die einzelnen \marg{Schlüssel} keine \oarg{Präfixe} oder \oarg{Suffixe} definieren kann.

Die andere Möglichkeit sieht vor, den Befehl \cs{cites} zu verwenden, 
bei dem für jeden Autor\,/\,jedes Werk sowohl \oarg{Präfixe} als \oarg{Suffixe} definiert werden kann.
Zudem lässt sich für die gesamte Reihung ein \oarg{Präfix} und ein \oarg{Suffix} festlegen:
\begin{lfgwcode}{label={lis:code:cites}}
\cites(Prä-Präfix)(Suf-Suffix)
  *@\oarg{Präfix}\oarg{Suffix}\marg{Schlüssel}@*%
  *@\oarg{Präfix}\oarg{Suffix}\marg{Schlüssel}@*%
  *@\oarg{Präfix}\oarg{Suffix}\marg{Schlüssel}\ldots@*
\end{lfgwcode}
\begin{lfgwexample}{label={lis:example:cites}}
Der öffentliche Raum ist Teil einer Stadt \cites(vgl.)(){Osland2016}{Evangelidis2014}.
\end{lfgwexample}
 
\minisec{\cs{parencite}}
Manchmal soll die Zitation in Klammern stehen.
Um dies nicht händisch in runde Klammern setzen zu müssen (und ggf. die \enquote{Klammerschachtelregel} zu verletzen),
kann dafür der Befehl  \cs{parencite} verwendet werden:
\begin{lfgwcode}{label={lis:code:parencite}}
\parencite*@\oarg{Suffix}\marg{Schlüssel}@*
\end{lfgwcode} 
Mit diesem Zitationsbefehl wird die korrekte Ordnung von korrespondierenden Klammern berücksichtigt.
\begin{lfgwexample}{label={lis:example:parencite}}
\enquote{Der öffentliche Raum ist Teil einer Stadt.} \parencite{Osland2016}
\end{lfgwexample}

\minisec{\cs{parencites}}
Ebenso lassen sich auch mehrere Zitationen mit Klammern umschließen.
Dies wird mittels \cs{parencites} umgesetzt:
\begin{lfgwcode}{label={lis:code:parencites}}
\parencites(Prä-Präfix)(Suf-Suffix)%
*@\oarg{Präfix}\oarg{Suffix}\marg{Schlüssel}@*%
*@\oarg{Präfix}\oarg{Suffix}\marg{Schlüssel}@*%
*@\oarg{Präfix}\oarg{Suffix}\marg{Schlüssel}\ldots@*
\end{lfgwcode}
\begin{lfgwexample}{label={lis:example:parencites}}
Der öffentliche Raum ist Teil einer Stadt.\parencites(s.)(){Osland2016}%
[vgl.][]{Evangelidis2014}.
\end{lfgwexample}

\minisec{\cs{textcite}}
Neben den bereits angeführten \cs{cite}-Befehlen gibt es eine dritte Möglichkeit der Zitationsangabe:
\cs{textcite} ist vor allem für die Fälle zu nutzen, 
bei denen der Autor\,/\,das Werk im Fließtext genannt sein soll, 
aber die weiteren Angaben (Publikationsjahr, Seitenangabe) nur in runden Klammern dahinter.
\begin{lfgwcode}{label={lis:code:textcite}}
\textcite*@\oarg{Suffix}\marg{Schlüssel}@*
\end{lfgwcode} 

\begin{lfgwexample}{label={lis:example:textcite}}
Der öffentliche Raum ist Teil einer Stadt, sagt \textcite{Osland2016}.
\end{lfgwexample}

\minisec{\cs{textcites}}
Wiederum können mehrere Autoren\,/\,Werke mittels \cs{textcites} gelistet werden:
\begin{lfgwcode}{label={lis:code:textcites}}
\textcites(Prä-Präfix)(Suf-Suffix)%
  *@\oarg{Präfix}\oarg{Suffix}\marg{Schlüssel}@*%
  *@\oarg{Präfix}\oarg{Suffix}\marg{Schlüssel}@*%
  *@\oarg{Präfix}\oarg{Suffix}\marg{Schlüssel}\ldots@*
\end{lfgwcode}
\begin{lfgwexample}{label={lis:example:textcites}}
Der öffentliche Raum ist Teil einer Stadt, sagen \textcites{Osland2016}%
[vgl.][]{Evangelidis2014}.
\end{lfgwexample}


\minisec{\cs{footcite}}
Darüberhinaus gibt es weitere \cs{cite}-Befehle, 
die die Einbettung der Zitation beeinflussen. 
Zunächst kann man mit \cs{footcite} die Zitation direkt als eigene Fußnote setzen:
 \begin{lfgwcode}{label={lis:code:footcite}}
\footcite*@\oarg{Präfix}\oarg{Suffix}\marg{Schlüssel}@*
\end{lfgwcode}
\begin{lfgwexample}{label={lis:example:footcite}}
\enquote{Der öffentliche Raum ist Teil einer Stadt.}\footcite{Osland2016}
\end{lfgwexample}
\cs{footcite} ist das Äquivalent zu \lstinline/\footnote{\cite{Osland2016}.}/
was jedoch manche (überflüssige) Tipparbeit spart.

\minisec{\cs{footcites} }
Für mehrere Autoren\,/\,Werke in einer Fußnote gibt es auch \cs{footcites}:
\begin{lfgwexample}{label={lis:example:footcites}}
\enquote{Der öffentliche Raum ist Teil einer Stadt.}\footcites(s.)(){Osland2016}%
[vgl.][]{Evangelidis2014}
\end{lfgwexample}
 
 
\minisec{\cs{smartcite}}
Eine clevere Art und Weise Zitationen als Fußnote zu setzen,
bietet der Befehl \cs{smartcite}.
 \cs{smartcite} reagiert auf die Umgebung des Befehls:
 Befindet sich \cs{smartcite} innerhalb des Fließtexts wird es wie \cs{footcite} behandelt 
 und die Zitation in eine Fußnote setzen. 
Wird die Zitation allerdings in einer Fußnote aufgerufen,
erfolgt die Ausgabe nach dem Schema von \cs{cite}. 
 is a clever 
\begin{lfgwcode}{label={lis:code:smartcite}}
\smartcite*@\oarg{Suffix}\marg{Schlüssel}%@*
\end{lfgwcode} 

\begin{lfgwexample}{label={lis:example:smartcite}}
Der öffentliche Raum ist Teil einer Stadt.\smartcite{Osland2016} 
Eventuell aber zugangsbeschränkt.\footnote{\smartcite[vgl.][]{Evangelidis2014}.}
\end{lfgwexample}


\minisec{\cs{smartcites}}
Wiederum gibt es analog auch den Befehl \cs{textcites}, 
um mehrere Autoren\,/\,Werke clever zu zitieren:
\begin{lfgwcode}{label={lis:code:smartcites}}
\smartcites(Prä-Präfix)(Suf-Suffix)%
  *@\oarg{Präfix}\oarg{Suffix}\marg{Schlüssel}@*%
  *@\oarg{Präfix}\oarg{Suffix}\marg{Schlüssel}@*%
  *@\oarg{Präfix}\oarg{Suffix}\marg{Schlüssel}\ldots@*
\end{lfgwcode}
\begin{lfgwexample}{label={lis:example:smartcites}}
Der öffentliche Raum ist Teil einer Stadt.\smartcites{Osland2016}{Evangelidis2014} 
Eventuell aber zugangsbeschränkt.\footnote{\smartcites{Osland2016}%
[cf.][]{Evangelidis2014}.}
\end{lfgwexample}

\minisec{\cs{autocite}}
Mit  \cs{autocite} ist eine individuelle und flexible Zitationsangabe möglich,
indem man in der Präambel steuert,
wie \cs{autocite} ausgegeben werden soll.
Für gewöhnlich stehen folgende Optionen zur Verfügung:
\begin{labeling}{footnote}
	\item[plain] Ausgabe wie \cs{cite}
	\item[inline]Ausgabe wie \cs{parencite}
	\item[footnote]Ausgabe wie \cs{footcite}
\end{labeling}
\begin{lfgwcode}{label={lis:code:autocite}}
\autocite*@\oarg{Präfix}\oarg{Suffix}\marg{Schlüssel}@*
\end{lfgwcode} 

\begin{lfgwexample}{label={lis:example:autocite}}
Der öffentliche Raum ist Teil einer Stadt \autocite{Osland2016} 
\end{lfgwexample}

\minisec{\cs{fullcite} \cs{footfullcite}}
Mit den Befehlen \cs{fullcite} und \cs{footfullcite} werden zwei Möglichkeiten gegeben,
mit denen man den kompletten Bibliographie-Eintrag in den Fließtext bzw. in die Fußnote schreiben kann.
\begin{lfgwcode}{label={lis:code:footfullcite}}
\fullcite*@\oarg{Präfix}\oarg{Suffix}\marg{Schlüssel}@*
\footfullcite*@\oarg{Präfix}\oarg{Suffix}\marg{Schlüssel}@*
\end{lfgwcode} 

\begin{lfgwexample}{label={lis:example:footfullcite}}
Der öffentliche Raum ist Teil einer Stadt.\footfullcite{Osland2016}
Das steht auch geschrieben bei \fullcite{Evangelidis2014}
\end{lfgwexample}



\minisec{\cs{citeauthor} \cs{citetitle}}
Neben den \enquote{geläufigen} \cs{cite}-Befehlen kann man auch nur den oder die Autoren zitieren 
und ebenso nur den Werktitel.
Dies funktioniert für den Fließtext und für Fußnoten gleichermaßen:
\begin{lfgwcode}{label={lis:code:citeauthor}}
\citeauthor *@ \oarg{Präfix}\oarg{Suffix}\marg{Schlüssel} @*
\end{lfgwcode} 
  and for the works 
\begin{lfgwcode}{label={lis:code:citetitle}}
\citetitle *@\oarg{Präfix}\oarg{Suffix}\marg{Schlüssel} @*
\end{lfgwcode} 

\begin{lfgwexample}{label={lis:example:citetitle}}
Der öffentliche Raum ist Teil einer Stadt sagt \citeauthor{Osland2016} in \citetitle{Osland2016}.
\footnote{Der öffentliche Raum ist Teil einer Stadt sagt \citeauthor{Osland2016} in \citetitle{Osland2016}.}
\end{lfgwexample}


\minisec{\cs{nocite}}
Möchte man einen Eintrag nicht als Zitation im Text haben, 
aber auf die Auflistung in der Bibliographie nicht verzichten,
dann kann man \cs{nocite}\marg{Schlüssel} verwenden.
Die Zitation wird damit \enquote{unsichtbar} zitiert:
Sie taucht nicht im Fließtext aber dennoch in der Bibliographie.




\section{Bibliografie-Stile}\label{sec:bibliografiestile}
Mit der zunehmenden Anwendung von \LaTeX{} in den Geisteswissenschaften steigt auch die Frage nach den \enquote*{passenden} Zitierstilen für die unterschiedlichen Fachgebiete:
Sei es Archäologie, Alte Geschichte, Filmwissenschaft, Theologie usw.
\subsection{Standardstile von biblatex}

biblatex (--> verschiedene biblatex-Stile mit deren Optionen erläutern, die für Geisteswissenschaftler relevant sind)
-- biblatex-dw (sehr viele Optionen)\\
-- archaeologie (sehr viele Optionen)\\
-- geschichtsfrkl (viele Optionen)\\
-- historische-zeitschrift\\
-- apa\\
-- historian\\
-- biblatex-fiwi\\
-- arthistory-bonn\\

\minisec{Nützliche Bibliografiestile, die nachinstalliert werden müssen}

Installation als Paket nötig:
...



\section{Ein Beispiel für Historiker: Quellen und Sekundärliteratur}\label{sec:bibliografie}
\index{Quellenverzeichnis}
\index{Bibliographie}
\index{Literaturverzeichnis}
\index{Sekundärliteratur}


%\begin{lfgwcode}{}  % läuft bei mir nicht - thm; ist auch nur für die Ausgabe einer Überschrift (ms))
%\printbibheading[%
%  heading=bibliography,% Standard
%  %heading=bibnumbered,% if you want it numbered
%  title={Bibliographie}]% Überschrift für Bibliographie  
%\end{lfgwcode}


\begin{lfgwcode}{label={lis:bibantikequellen}}
\printbibliography[%
  heading=subbibliography,
  %heading=subbibnumbered,% if you want it numbered
  keyword=ancient,%
  title={Antike Quellen}]
\end{lfgwcode}

\begin{lfgwcode}{label={lis:bibabksekliteratur}}
\printbibliography[%
  heading=subbibliography,
  keyword=corpus,%
  title={Abkürzungen und Sigel}]

\printbibliography[%
  heading=subbibliography,
  notkeyword=ancient,%
  notkeyword=corpus,%
  title={Sekundärliteratur}]
\end{lfgwcode}

