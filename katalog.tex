% !TeX root = lfgw.tex
\iffalse
%Text in den Vorspann
%----------------------
\newenvironment{bsp}{\begin{framed}\begin{footnotesize}}%
{\end{footnotesize}\end{framed}}
%----------------------
\newcommand\catalogueentry[1]{%
\RaggedRight\begingroup
\setkeys{catalogue}{#1}%
\ifdef{\KVhouse}{\section{\KVhouse
			\ifdef{\KVlabel}{\label{\KVlabel}}{}}
			}{}%
\begin{labeling}{Beschreibung}
\ifdef{\KVdescription}{\item[Beschreibung] \KVdescription}{}%
\ifdef{\KVlocation}{\item[Verortung] \KVlocation}{}%
\ifdef{\KVinterior}{%
	\item[Ausstattung] \KVinterior %
	\ifboolexpr{bool{@KVinteriorM} or bool {@KVinteriorW} or bool {@KVinteriorS}}{%
	\begin{labeling}{Wandgemälde}
			\ifdef{\KVinteriorM}{\item[Mosaike:] \KVinteriorM}{}
			\ifdef{\KVinteriorW}{\item[Wandgemälde:] \KVinteriorW}{}
			\ifdef{\KVinteriorS}{\item[Statuen:] \KVinteriorS}{}
			\end{labeling}
		}{}}%
	{%
	\ifboolexpr{bool{@KVinteriorM} or bool {@KVinteriorW} or bool {@KVinteriorS}}{%
		\item[Ausstattung]%
	\begin{labeling}{Wandgemälde}	
				\ifdef{\KVinteriorM}{\item[Mosaike:] \KVinteriorM}{}
				\ifdef{\KVinteriorW}{\item[Wandgemälde:] \KVinteriorW}{}
				\ifdef{\KVinteriorS}{\item[Statuen:] \KVinteriorS}{}
			\end{labeling}
	}{}}%
\ifdef{\KVsize}{	\item[Größe] \SI{\KVsize}{\meter\squared}}{}%
\ifdef{\KVbackref}{\item[Erwähnungen]S.\ \KVbackref}{}%		oder \pno{}
\end{labeling}
\endgroup
}

\makeatletter
\newbool{@KVinteriorM}
\newbool{@KVinteriorW}
\newbool{@KVinteriorS}
\define@key{catalogue}{house}{\def\KVhouse{#1}}
\define@key{catalogue}{label}{\def\KVlabel{#1}}
\define@key{catalogue}{description}{\def\KVdescription{#1}}
\define@key{catalogue}{location}{\def\KVlocation{#1}}
\define@key{catalogue}{size}{\def\KVsize{#1}}
\define@key{catalogue}{interior}{\def\KVinterior{#1}}
\define@key{catalogue}{interiorM}{\def\KVinteriorM{#1}\booltrue{@KVinteriorM}}
\define@key{catalogue}{interiorW}{\def\KVinteriorW{#1}\booltrue{@KVinteriorW}}
\define@key{catalogue}{interiorS}{\def\KVinteriorS{#1}\booltrue{@KVinteriorS}}
\define@key{catalogue}{backref}{\def\KVbackref{#1}}
\makeatother
\fi
%+++++++++++++++++++++++++++
\chapter{Kataloge erstellen}
Manche und vor allem größere und damit umfangreichere Textdokumente enthalten am Ende der Arbeit einen Katalog, 
in dem die untersuchten Materialien in einem bestimmten System aufgeschlüsselt präsentiert werden.
Ein solcher Katalog kann eine große Bandbreite von bspw.  Bildern, Statuen oder Inschriften umfassen.\footnote{Bezug auf DTK 3/2016}

Eine händische Erstellung der einzelnen Einträge bspw. über \cs{section} oder \cs{subsection} 
und anschließend in einer {items}-Umgebung ist nicht effizent, fehleranfällig und nur bei wenigen Katalogeinträgen einsetzbar.
Es muss zudem berücksichtigt werden, dass  die vorgegebenen Kategorien nicht für jeden Katalogeintrag passend sind, 
sodass  Kategorien leer bleiben  und dann im Katalogeintrag nicht auftauchen sollen.
Dabei soll der Code des Katalogeintrags möglichst viel redundante Tipp-Arbeit abnehmen, wie sie bspw. bei Maßeinheiten vorkommt.
Darüber hinaus sollen alle Einträge immer gleich formatiert sein und ihr Aussehen global verändert werden können.

Als Aufgabe soll ein Katalog zu den Häusern in Pompeji angelegt werden,
der Auskunft über den Namen des Hauses, dessen Verortung und Grundstücksgröße  geben soll.
Zudem soll eine kurze Beschreibung enthalten sein und die Angabe über die Innenausstattung, 
die wiederum auf die Untergruppen Mosaik, Wandgemälde und Skulptur aufgeschlüsselt werden kann. 
Zum Schluss soll im Katalogeintrag angezeigt werden, 
auf welchen Seiten des Fließtextes das Haus genannt wird.

Die gefundene Lösung für die Umsetzung der Anforderungen funktioniert mit Hilfe des   \Package{keyval}-Pakets.\footnote{Basierend auf der Idee vorgestellt auf: \url{http://tex.stackexchange.com/a/254336/98739}}
Dafür werden in der Präambel verschiedene Schlüssel (\cs{keys}) definiert.
In der Grundversion sieht die Definition eines Eintrags wie folgt aus:\footnote{Vgl. \url{http://www.tug.org/tugboat/tb30-1/tb94wright-keyval.pdf}}
\begin{lstlisting}
\define@key{family}{key}{#1}
\end{lstlisting}
Für das konkrete Beispiel wird die \cs{key}-Familie (\cs{family}) mit \cs{catalogue} angegeben,
der Schlüssel (\cs{key}), was einer Kategorie im Katalog entspricht, 
als \cs{house} bezeichnet und als Resultat soll der Wert im Makro \cs{KVhouse} gespeichert werden:
\begin{lstlisting}[]
\define@key{catalogue}{house}{\def\KVhouse{#1}}
\end{lstlisting}
Dem Beispiel entsprechend können alle Kategorien als Schlüssel angelegt werden (\cref{lis:cat-key}):
\begin{lstlisting}[caption={Definition der Schlüssel},label={lis:cat-key}]
\makeatletter
\define@key{catalogue}{house}{\def\KVhouse{#1}}
\define@key{catalogue}{label}{\def\KVlabel{#1}}
\define@key{catalogue}{description}{\def\KVdescription{#1}}
\define@key{catalogue}{location}{\def\KVlocation{#1}}
\define@key{catalogue}{size}{\def\KVsize{#1}}
\define@key{catalogue}{interior}{\def\KVinterior{#1}}
\define@key{catalogue}{interiorM}{\def\KVinteriorM{#1}}
\define@key{catalogue}{interiorW}{\def\KVinteriorW{#1}}
\define@key{catalogue}{interiorS}{\def\KVinteriorS{#1}}
\makeatother
\end{lstlisting}
Das Aussehen eines Katalogeintrages wird separat mit dem Makro
 \cs{newcommand}\cs{catalogueentry}[1]{\ldots} 
definiert:
Darin wird zunächst festgelegt,
dass eine neue Gruppe beginnt (\cs{begingroup}), sodass
es zu keinen Problemen mit den jeweils definierten Schlüsseln kommt, 
da diese für jeden Katalogeintrag neu definiert werden.
Dann sollen die  Einträge im Flattersatz gesetzt werden (\cs{RaggedRight}) und schließlich die Angabe,
 welche Schlüsselfamilie (hier \cs{catalogue}) auszulesen ist.
\begin{lstlisting}[caption={Definition der Katalogeinträge, Anfang},label={lis:cat-1}]
\newcommand\catalogueentry[1]{%
\begingroup
\RaggedRight
\setkeys{catalogue}{#1}
...
\end{lstlisting}
Es folgt in der Definition die Verarbeitung der einzelnen Schlüssel:
Da nur die Ausgabe einer Kategorie erfolgen soll,
wenn diese auch mit Informationen versehen ist,
wird dies über die Abfrage \cs{ifdef} erledigt.
Der Inhalt der Kategorie \cs{house} wird als \cs{section}-Titel verwendet und, wenn vorhanden, mit einem \cs{label} versehen.
Die weiteren Kategorien sollen in einer {labeling}-Umgebung aufgelistet werden.
Die Definition wird mit \cs{endgroup} geschlossen.
%[caption={Definition der Katalogeinträge, Fortsetzung},label={lis:cat-2}]
\begin{lfgwcode}{label={lis:cat-2}}
...
\ifdef{\KVhouse}{\section{\KVhouse
  \ifdef{\KVlabel}{\label{\KVlabel}}{}}}{}
\begin{labeling}{Beschreibung}
  \ifdef{\KVdescription}{\item[Beschreibung] \KVdescription}{}
  \ifdef{\KVlocation}{\item[Verortung] \KVlocation}{}
  \ifdef{\KVinterior}{\item[Ausstattung] \KVinterior
    \begin{labeling}{Wandgemälde}
      \ifdef{\KVinteriorM}{\item[Mosaike:] \KVinteriorM}{}
      \ifdef{\KVinteriorW}{\item[Wandgemälde:] \KVinteriorW}{}
      \ifdef{\KVinteriorS}{\item[Statuen:] \KVinteriorS}{}
    \end{labeling}
  }{}
  \ifdef{\KVsize}{\item[Größe] \SI{\KVsize}{\meter\squared}}{}
\end{labeling}
\endgroup
}
\end{lfgwcode}

Mit dieser Einstellung lassen sich die Katalogeinträge schon sehr gut darstellen.
Im Fließtext des Hauptdokuments kann man an gewünschter Stelle mit \cs{catalogueentry} einen Katalogeintrag eintragen.
So wird aus den folgenden Angaben ein Katalogeintrag, der so aussieht:
\begin{lstlisting}
    \catalogueentry{%
        house={Haus des M. Fabius Rufus},
        label={haus:M-Fabius-Rufus},
        size={172},
        description={Haus besteht aus mehreren Einzelgebäuden.},
        location={Regio VII, Insula 16, Eingang 17--22.},
        interior={Reicher Fundkomplex.},
        interiorM={S/W-Mosaik.},
        interiorW={Dionysius mit einer Mänade, Narzissus und ein Cupido, Hercules und Deinira etc.},
        interiorS={Bronzene Statue eines Epheben.},
    }
\end{lstlisting}

\begin{lfgwprint}{house1}
\catalogueentry{%
	house={Haus des M. Fabius Rufus},
	label={haus:M-Fabius-Rufus},
	size={172},
	description={Haus besteht aus mehreren Einzelgebäuden.},
	location={Regio VII, Insula 16, Eingang 17--22.},
	interior={Reicher Fundkomplex.},
	interiorM={S/W-Mosaik.},
	interiorW={Dionysius mit einer Mänade, Narzissus und ein Cupido, Hercules und Deinira etc.},
	interiorS={Bronzene Statue eines Epheben.},
}
\end{lfgwprint}
Wie man bei diesem Beispiel sieht, ist die Reihenfolge, in der die Kategorien
 angegeben werden, irrelevant, da die Definition in der Präambel entscheidend ist.
Der Wert bei \cs{size} wird intern sogleich an das vordefinierte \cs{SI}-Makro mit entsprechender Einheit (\si{\meter\squared}) übergeben.
Ähnlich kann auch mit anderen Angaben verfahren werden (bspw. bei Abbildungen können die \cs{label} an ein vordefiniertes Macro \cs{cref} übergeben werden).

Bei dem oben beschriebenen Katalogeintrag zum 
›Haus des M.~Fabius Rufus‹ sind alle Kategorien ausgefüllt. 
Wenn eine Kategorie nicht ausgefüllt ist, wird sie nicht ausgegeben.
Allerdings besteht die Kategorie \cs{interior} [Ausstattung]  zusätzlich aus drei Unterkategorien 
(\cs{interiorM} [Mosaike], \cs{interiorW} [Wandgemälde], \cs{interiorS} [Statuen]).
Diese sollen auch dann angezeigt werden, wenn \cs{interior} selbst nicht definiert ist.

Ein solcher Fall tritt beim ›Haus des Wilden Ebers‹ auf. {Haus-des-Wilden-Ebers}
Im Fließtext ist der Katalogeintrag wie folgt ausgefüllt (\cref{lis:cat-eber}):
\begin{lstlisting}[caption={Katalogeintrag für das Haus des Wilden Ebers},label={lis:cat-eber}]
\catalogueentry{%
	house={Haus des Wilden Ebers},
	label={Haus-des-Wilden-Ebers},
	size={54},
	description={Renovierung nach Erdbeben 62\,n.\,Chr.},
	location={Regio VII, Insula 4, Eingang 48, 43.},
	interiorM={S/W-Mosaik.},
	interiorW={Venus, Leda und der Schwan, Ariadne und Theseus.},
}
\end{lstlisting}
So wird daraus:{Haus-des-Wilden-Ebers}
\begin{bsp}
\catalogueentry{%
	house={Haus des Wilden Ebers},
	label={Haus-des-Wilden-Ebers},
	size={54},
	description={Renovierung nach Erdbeben 62\,n.\,Chr.},
	location={Regio VII, Insula 4, Eingang 48, 43.},
	interiorM={S/W-Mosaik.},
	interiorW={Venus, Leda und der Schwan, Ariadne und Theseus.},
}
\end{bsp}
Um zu diesem Ergebnis zu kommen, 
ist es notwendig, mit Booleschen Operatoren zu arbeiten.
Dafür müssen bei den \cs{key}-Definitionen drei Operatoren eingeführt werden (\cref{lis:cat-bool}):
\begin{lstlisting}[,caption={Definition der Booleschen Operatoren},label={lis:cat-bool}]
\newbool{@KVinteriorM}%Mosaik
\newbool{@KVinteriorW}%Wandgemälde
\newbool{@KVinteriorS}%Statue
\end{lstlisting}
Diese Booleschen Operatoren werden bei der Definition der einzelnen Einträge eingebaut und als 
\cs{true} gesetzt, wenn dieser Eintrag mit Informationen versehen wird.
Konkret sieht die Modifikation so aus (\cref{lis:cat-bool2}):
 \begin{lstlisting}[,caption={Einsetzten der Booleschen Operatoren in die Schlüssel},label={lis:cat-bool2}]
\define@key{catalogue}{interiorM}{\def\KVinteriorM{#1}\booltrue{@KVinteriorM}}
\define@key{catalogue}{interiorW}{\def\KVinteriorW{#1}\booltrue{@KVinteriorW}}
\define@key{catalogue}{interiorS}{\def\KVinteriorS{#1}\booltrue{@KVinteriorS}}
\end{lstlisting}
Zudem müssen die Booleschen Operatoren bei der Ausgabe der Katalogeinträge eingesetzt werden,
sodass geprüft wird (\cs{ifboolexpr}), ob einer oder mehrere der Operatoren auf \cs{true} gesetzt ist (\cref{lis:cat-bool3}).
\begin{lstlisting}[,caption={Einsetzten der Booleschen Operatoren in den Katalogeintrag},label={lis:cat-bool3}]
\ifdef{\KVinterior}{%
  \item[Ausstattung] \KVinterior 
  \ifboolexpr{bool{@KVinteriorM} 
    or bool{@KVinteriorW} 
    or bool{@KVinteriorS}}{%
      \begin{labeling}{Wandgemälde}
      \ifdef{\KVinteriorM}{\item[Mosaike] \KVinteriorM}{}
      \ifdef{\KVinteriorW}{\item[Wandgemälde] \KVinteriorW}{}
      \ifdef{\KVinteriorS}{\item[Statuen] \KVinteriorS}{}
    \end{labeling}
    }{}}
  {\ifboolexpr{bool{@KVinteriorM} 
    or bool{@KVinteriorW} 
    or bool{@KVinteriorS}}{%
  \item[Ausstattung]%
  \begin{labeling}{Wandgemälde}	
  \ifdef{\KVinteriorM}{\item[Mosaike] \KVinteriorM}{}
    \ifdef{\KVinteriorW}{\item[Wandgemälde] \KVinteriorW}{}
    \ifdef{\KVinteriorS}{\item[Statuen] \KVinteriorS}{}
  \end{labeling}
	}{}}
\end{lstlisting}
Mit dieser modifizierten Ergänzung der Katalogeintragsdefinition werden die Unterkategorien von \cs{interior} ausgegeben, 
auch wenn \cs{interior} selbst nicht definiert ist.