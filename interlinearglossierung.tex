\chapter{Zusammenhängende Texte parallelisieren}

\section{Interlinearglossierung}
\autor{Thomas Meyer}

\index{Interlinearübersetzung}

ACHTUNG: Der folgende Text wird auf das Paket \paket{cgloss4e} umgestellt!

Das Paket \paket{covington} von Michael Covington stellt einige primär für Linguisten nützliche
Befehle zur Verfügung. Mit seiner Hilfe ist es relativ einfach, eine einfache 
Interlinearglosse zu erstellen, d.\,h. Wort für Wort zu übersetzen oder zu kommentieren:

\begin{lstlisting}
\gll La  neige,  qui {n'a pas} cesse       de tomber 
     Der Schnee, der {nicht hat} aufgehört zu fallen 
\glt Der Schnee, der nicht zu fallen aufgehört hat
\glend 
\end{lstlisting}
stand oder
%\gll La  neige,  qui {n'a pas} cesse       de tomber 
%     Der Schnee, der {nicht hat} aufgehört zu fallen 
%\glt Der Schnee, der nicht zu fallen aufgehört hat
%\glend

Zu beachten ist, dass es neben dem Kommando \lstinline/\gll/ auch ein Kommando 
\lstinline/\glll/ gibt, das die Zusammenstellung von drei Zeilen ermöglicht.

\textbf{Achtung!} Die neueste Version von \paket{covington} stammt aus dem Jahr 
2001 und verwendet altertümliche Fontbefehle. Damit das Paket mit modernen Dokumentklassen 
wie \KOMAScript\ zusammenarbeitet, war es nötig, im Paket die Zeilen 225-227 auszukommentieren.
