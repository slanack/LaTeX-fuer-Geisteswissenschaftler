\chapter{Diagramme zeichnen}
\autor{Thomas Hilarius Meyer}

Unter \TeX\ stehen grundsätzlich verschiedene Ansätze zur Verfügung. 
Man kann allgemeine und spezielle Lösungen unterscheiden:
Erstere eignen sich für verschiedenste konkrete Anwendungen, überlassen aber wegen ihrer Abstraktheit 
und Allgemeinheit dem Benutzer viel Entscheidungsfreiheit -- und Konfigurationsaufwand. 
Demgegenüber gibt es spezielle Lösungen für einzelne konkrete Aufgaben in Form von kleinen Paketen,
die mit wenig Aufwand die Erstellung etwa einer Zeitleiste oder einer Mindmap ermöglichen.
Für andere Aufgaben sind diese Pakete nicht geeignet.

Weil sich diese Anleitung nicht in erster Linie an Grafiker wendet, sondern an Geisteswissenschaftler,
die im Rahmen ihrer Arbeit eher ab und zu ein Diagramm benötigen, werden im folgenden primär
kleine spezialisierte Pakete für konkrete Aufgaben vorgestellt: 
für linguistische Strukturen,
Baumdiagramme,
genealogische Stammbäume,
Mindmaps,
Zeitleisten 
und zur Visualisierung von statistischen Daten in Form von Torten-, Balken und Liniendiagrammen.

Davor wird sehr kurz auf die diesen Spezialpaketen zugrunde liegenden allgemeinen Grafiklösungen
von \TeX\ -- \METAPOST, PSTricks und \TikZ{} -- eingegangen. Wer sich stark für die Erzeugung von Grafiken
mit \TeX\ interessiert, sollte sich mit diesen näher beschäftigen.

Zu Beginn wird jedoch noch eine vielleicht verpönte, im Alltag aber nich gerade seltene Möglichkeit
der Diagrammeinbindung in \LaTeX\ eingegangen: 

\section{Der unsaubere Weg: Externe Programme benutzen}
Es existieren zahlreiche Programme außerhalb der \TeX-Welt, mit denen sich Diagramme erstellen 
lassen: Statistische Daten in Tabellen lassen sich mit Excel und Libre Office in Säulen- und 
Tortendiagramme verwandeln, daneben gibt es Spezialprogramme zum Zeichnen von Mindmaps
und schließlich lassen sich Organigramme, Zeitleisten etc. mit Hilfe von CAD-Programmen o.\,ä.
erstellen.

Alle diese Lösungen können zur Zusammenarbeit mit \LaTeX\ genutzt werden, wenn sie nur in der 
Lage sind, das erarbeitete Diagramm als Grafik in einem gängigen Format zu speichern.

Auf der \LaTeX-Ebene bleibt dann nur noch das Einbinden der fertigen Abbildung als Grafik zu
tun. Das Verfahren wird in Abschnitt~\ref{grafik-einbinden} auf Seite~\pageref{grafik-einbinden}
beschrieben.

Das Verfahren mag einem \TeX-Puristen unsauber erscheinen, wird aber häufig genutzt.
M.\,E. ist die konzeptionelle Offenheit und Modularität von \TeX\ ein Vorzug, zu dem man stehen sollte.


\section{Die drei grundsätzlichen Wege: \METAPOST, PSTricks und \TikZ}

Es gibt unter \TeX\ drei grundsätzlich verschiedene Lösungen zur Erzeugung von Grafiken;
alle drei Ansätze haben Vorzüge und Nachteile und können auch nebeneinander verwendet werden:
\METAPOST, PSTricks und \TikZ. Auf diesen Methoden setzen die spezialisierten Pakte auf.

\minisec{Metapost}
Bei \METAPOST{} handelt es sich um eine eigene deklarative Programmiersprache zur Definition
von Grafiken, die aus den MP-Befehlen eine \PS-Datei erzeugt, die in einem \LaTeX-Dokument
eingebunden werden kann.
Es wurden spezifische Pakete entwickelt, um diesen Zwischenschritt abzukürzen und 
\METAPOST-Code direkt in \LaTeX -Dokumente einbinden zu können.

Zu \METAPOST{} existiert eine ausführliche Beschreibung in der Dante-Reihe;%
\footcite{entenmann:metapost}
außerdem gibt es große Mengen an online-Dokumentation.

Wer sich ausführlich mit \METAPOST{} beschäftigt, erschließt sich eine ganze Welt unbegrenzter
grafischer Möglichkeiten. Für die bescheidenen Zwecke der meisten geisteswissenschaftlichen
\TeX-Benutzer ist jedoch der Lernaufwand im Vergleich zu den fertig vorkonfigurierten Spezialpaketen,
die im folgenden vorgestellt werden, zu groß.


\minisec{PSTricks}
\paket{PSTricks} geht einen anderen Weg: Es handelt sich hier um eine Sammlung von Makros, die die
Fähigkeiten von Postscript als eigenständiger Programmiersprache ausnutzt und zur komfortablen
Nutzung aus \LaTeX{} heraus erschließt.

Das hat ausgesprochene Vorteile: \PS{} hat z.\,B. wesentliche bessere Möglichkeiten zur
effizienten und genauen Durchführung von Berechnungen, als \LaTeX{}.
Somit eignet sich PSTricks vor allem zur Auswertung und grafischen Darstellung auch sehr komplexer
mathematischer Funktionen und großer Mengen von statistischen Daten. 

Der Preis der Verwendung von \PS{} ist, dass dann auch eine .ps-Datei erzeugt werden muss,
d.\,h. statt der direkten Ausgabe einer PDF-Datei (die ja in aller Regel das Ziel ist), muss zuerst
eine DVI-Datei erzeugt werden, die dann mittels des Kommandozeilentools \prog{dvips} in eine
Postscript-Datei konvertiert wird. Aus dieser wird dann mittels \prog{ps2pdf} die
angezielte PDF-Datei.

Naturgemäß bedingt der riesige Leistungsumfang von PSTricks eine große Komplexität; wieder handelt es sich
um eine ganz eigene Welt, die zu erschließen ist.
In der Dante-Reihe existiert eine umfangreiche Monographie von Herbert Voß zu PSTricks.%
\footcite{voss:pstricks}

Ähnlich wie schon bei \METAPOST{} ist es auch bei PSTricks so, dass im geisteswissenschaftlichen
Alltagseinsatz nur ein winziger Bruchteil der Funktionalität zum Einsatz kommen wird.

Auch ist es nicht notwendig, das Gesamtsystem von PSTricks zu meistern, denn es exisiteren eine
ganze Reihe von Spezialpaketen, die teilweise sehr einfach einzusetzende Lösungen für
Spezialanwendungen anbieten.
\Cref{tab:PSTricksPakete} bietet eine Zusammenstellung solcher PSTricks-Pakete, die für Historiker oder
Philologen von besonderem Interesse sein könnten.

\begin{table}
    \begin{center}
    \begin{tabular}{ll}
        Aufgabe &					Paket \\
        \hline
        Linguistische Strukturen & 		\paket{pst-jtree}\\
        Baumdiagramme &					\paket{pst-tree}\\
        Organigramme &					\paket{pst-node}\\
        Torten- und Säulendiagramme &	\paket{pst-bar}\\
        Liniendiagramme &				\paket{pst-plot}\\
    \end{tabular}
    \caption{Empfehlenswerte PSTricks-Pakete für geisteswissenschaftliche Bedürfnisse}
    \label{tab:PSTricksPakete}
    \end{center}
\end{table}

Neben der Monographie von Herbert Voß sei zu ihrem Einsatz auf die in der Regel sehr
guten Paketdokumentationen verwiesen.%
\footcite[][Außerdem immer noch lesenswert:]{roemer:dtk2008}


\minisec{\TikZ}
\enquote{Mit dem Paket \paket{pgf} (portable graphic format) hat Till Tantau eine Sammlung von Makros
    zur Verfügung gestellt, die eine Einbettung von grafischen Elementen im normalen Text erlauben,
    bei der Übersetzung aber nicht, wie beispielsweise \METAPOST{} [\ldots] oder PSTRICKS 
    [\ldots], auf externe Zwischenschritte angewiesen ist.
    \paket{pgf} ist eine reine PDF-Lösung, sodass eine Anwendung mit \pdfTeX{} möglich ist.
    Dies ist zum einen ein großer Vorteil bei der Erstellung von PDF-Dokumenten, aber zum anderen
    ein großer Nachteil, denn die rechnerischen Fähigkeiten des PDF-Formats sind faktisch gleich Null.
    Daher müssen alle Rechenoperationen auf \TeX-Ebene erfolgen, was bezüglich der Genauigkeit und der
    Effizienz ein Problem sein kann. 
    
    [\ldots]
    
    Die Syntax von \paket{pgf} ist stark an dem Ausgabeformat PDF orientiert, weshalb Till Tantau
    ein darauf aufbauendes Paket \paket{tikZ} (\paket{tikZ} ist kein Zeichenprogramm) entwickelt hat,
    welches die Befehlssyntax anwenderfreundlich gestaltet.}%
\footcite[767]{voss:einfuehrung}  

Ähnlich wie \METAPOST{} und PSTricks ist \paket{tikZ} ungemein umfangreich und erlaubt dem
geübten Benutzer praktisch \emph{alles}, zumal im geisteswissenschaftlichen Umfeld die Einschränkungen 
im mathematischen Bereich wenig ins Gewicht fallen.    
Und auch \paket{tikZ} erfordert vom Benutzer ein großes Maß an Lernbereitschaft und Einarbeitungszeit,
um den ganzen Leistungsumfang zu erschließen.


\section{Konkrete Lösungen}
Deshalb werden im folgenden eine Reihe von Paketen aufgebaut, die -- aufbauend auf \paket{tikZ}%
\footnote{Das ist eine willkürliche Entscheidung, die kein Qualitätsurteil über die drei Ansätze
    aussagen soll. Maßgeblich für die Entscheidung war, dass \paket{tikZ}-Lösungen problemlos
    mit PDF-Engines zusammenarbeiten und dass im geisteswissenschaftlichen Bereich die mangelnde
    Präzision und Geschwindigkeit komplexer Berechnungen keine Rolle spielen.
    Außerdem wird es in der Zukunft vermutlich so sein, dass Berechnungen innerhalb von 
    \LuaLaTeX{} auf
    der Lua-Ebene erledigt werden, so dass diese Einschränkung immer weniger eine Rolle spielen wird.
    Dennoch: bei intensiverer Arbeit mit Grafiken lohnt es sich auf jeden Fall, sich etwa mit den
    in Tabelle~\ref{tab:PSTricksPakete} genannten PSTricks-Paketen genauer zu beschäftigen.}
-- mit wenig Aufwand eine akzeptable Lösung für ausgewählte Probleme aus dem geisteswissenschaftlichen
Aufgabenbereich anbieten.


\subsection{Konstituentenstrukturen}
\label{linguistische-strukturen}
\index{Linguistik} \index{x-bar-Schema} \index{Phrasenstruktur} \index{Konstituentenstruktur}
\index{Baumdiagramm} \index{Stemma}

\autor{Christine Römer}

Hier geht es um graphische Darstellungen etwa im Kontext der Kontituentenstrukturanalyse.
Für die Wiedergabe von Beispielsätzen etc. vgl. den Abschnitt~\ref{linguistische-beispiele} auf Seite~\pageref{linguistische-beispiele}.

%% lfgw: 2.21 Linguistische Beispiele
\subsubsection{Einordnung}

Konstituentenanalysen von Wörtern und Sätzen vorzunehmen, sind in der Linguistik vom amerikanischen Strukturalismus entwickelte Verfahren der Segmentierung zur Ermittlung von syntaktischen Konstituentenhierarchien 
und -kategorien, die für die Beschreibung und Analyse von Sprachen relevant sind. 
Die Ergebnisse der Zerlegungen werden in Baumstrukturen (s. Beispiel~\ref{forestexample:1}) und  Klammerschreibungen (s. Beispiel~\ref{forestexample:2})
veranschaulicht. Noam Chomsky, der Kopf der Generativen Grammatik, zeigt in Form der X-bar-Theorie
bspw. in \enquote{Rules and Representations} die Beziehung zwischen Tense und dem COMP  folgendermaßen auf (S.\,170f.)

\begin{lfgwexample}{label={forestexample:1}}
\begin{forest}
[S´
  [COMP [$ \begin{Bmatrix}\emph{that} \\ \emph{for} \end{Bmatrix} $, align=center, base=bottom]]
  [S
   [NP [\emph{John}, tier=word]]
   [ $ \begin{Bmatrix} \emph{Tense}\\ \emph{to} \end{Bmatrix} $, align=center, base=bottom]
   [VP [\emph{leave}, tier=word]]
  ] ]
\end{forest}
\end{lfgwexample}

\begin{lfgwprint}{label={forestexample:2}}
\begin{center}
it is unclear $[_\text{S´} [_{\text{COMP}} \enspace \textnormal{what} ]
 [_\text{S}  \enspace  \textnormal{NP} \enspace \begin{Bmatrix} \emph{Tense}\\ \emph{to} \end{Bmatrix} \enspace \textnormal{VP}] ]$
\end{center}
\end{lfgwprint}

\lstset{frame=single}
\begin{lstlisting}
\begin{center}
it is unclear $[_\text{S´} [_{\text{COMP}} \enspace \textnormal{what} ]
 [_\text{S}  \enspace  \textnormal{NP} \enspace \begin{Bmatrix}{c}\emph{Tense}\\ \emph{to} \end{Bmatrix} \enspace \textnormal{VP}] ]$
\end{center}
\end{lstlisting}


\subsubsection{Baumstrukturen mit \texttt{forest}}

Sa\v{s}o \v{Z}ivanovi\'{c}, der Autor von \Package{forest}, schreibt in
\cite[3]{zivanovic:forest},
bezüglich Forest: \\
"`I believe, the most flexible tree typesetting package for \LaTeX\ you can get."'

Auf jeden Fall ist es einfacher als andere Pakete\footnote{Zu anderen
einschlägigen Paketen siehe \cite{roemer:dtk2016}.} zu handhaben, da es die Spreizung
der Äste entsprechend der Label an den Astenden selbstständig anpasst. Dabei bemüht es sich,
um schlanke Bäume. Auch komplexe Strukturen führen nicht zu Problemen. Durch die Anbindung
an PGF/TikZ können Ausschmückungen vorgenommen werden.

\minisec{Funktionsweise}

Das Paket wird mit \verb|\usepackage{forest}| in der Präambel geladen. Spezielle Bibliotheken
können mit ihrem Namen in der Regel als fakultatives Argument hinzugefügt werden: 
\verb|\usepackage[library name]{forest}|; beispielsweise diejenige für linguistische Strukturen
mit \verb|\usepackage[linguistics]{forest}|. Mit dieser Option werden die Strukturbäume
standardmäßig linksverzweigt.

Phonologische Strukturen (wie in Beispiel~\ref{forestexample:3}) sind in der \texttt{linguistik}-Bibliothek
in dem \texttt{GP1}-Stil angelegt. Der Stilname wird vor die erste sich öffnende Klammer des Baums 
geschrieben\footnote{\cite[Beispiel\,6]{zivanovic:forest}}.

\begin{lfgwexample}{label={forestexample:3}}
\begin{forest} GP1 [
  [O[x[f]][x[r]]]
  [R[N[x[o]]][x[s]]]
  [O[x[t]]]
  [R[N[x]]]
]
\end{forest}
\end{lfgwexample}

Die Baumstrukturen werden in die \texttt{forest-Umgebung} oder in den Befehl \verb|Forest*{   }|
mittels Umklammerungen der Konstituenten eingefügt. Die Kinder eines Knotens befinden sich 
innerhalb des Mutterknotens.\footnote{Den linguistischen Konstruktionen liegt die Generative
Grammatik zugrunde. So werden bei den Barstufen der Kategorien die Markierungen rechts ausgedruckt,
obwohl sie links eingegeben werden (siehe Beispiel~\ref{forestexample:4}).}

\begin{lfgwexample}{label={forestexample:4}}
\begin{minipage}{.3\linewidth}
\begin{forest}
[VP
  [DP]
  [V’
   [V]
  [DP]
  ]
]
\end{forest}
\end{minipage} \quad
\begin{minipage}{.2\linewidth}
\Forest*{
[VP
  [DP]
  [V’
   [V]
  [DP]
  ]
]
}
\end{minipage}
\end{lfgwexample}

Als letzte Labels können auch natürlichsprachliche Ausdrücke (\verb|tier=word|) eingefügt werden, 
die dann entsprechend den Konventionen kursiv geschrieben werden können, mit dem entsprechenden
Befehl. Wie im folgenden Beispiel (\ref{forestexample:5}) beim obersten Knoten zusehen ist, 
können Labels auch ohne Kantenverbindungen untereinander
geschrieben werden (\verb|align=center, base=bottom|).


\begin{lfgwexample}{label={forestexample:5}}
\begin{forest}
where n children=0{tier=word}{}
[Wort\\Stamm, align=center, base=bottom
[BM [\emph{Not}, tier=word]]
[Stamm
[Stamm [WBM [Präf [\emph{auf}]]] 
       [Wu [BM$'$ F [\emph{nahm e}]]]]
[WBM [Suff [\O]]]]       
]
\end{forest}
\end{lfgwexample}

Die Linien zu den Konstituenten können auch gestrichelt, gepunktet oder als Pfeile
gesetzt werden (Beispiel~\ref{forestexample:6}).

\begin{lfgwexample}{label={forestexample:6}}
\begin{forest} for tree={grow'=0,l=2cm,anchor=west,child anchor=west},
[Äste,rotate=90
  [normale Linie,edge label={node[midway,left,
        font=\scriptsize]{default}}]
  [keine Linie,no edge]
  [gepunktete Linie,edge=dotted]
  [gestrichelte Linie,edge=dashed]
  [gestrichelte rote Linie,edge={dashed,red}]
  [Pfeil, edge={->, very thick}]
]
\end{forest}
\end{lfgwexample}

Die Bibliothek \texttt{edges} ermöglicht verzweigte Kanten/Äste bei horizontal wachsenden
Strukturen. Sie wird mit \verb|\useforestlibrary{edges}| geladen. Man kann damit beispielsweise
Begriffshierarchien aufzeigen (wie in \ref{forestexample:7}, \ref{forestexample:8}).

\begin{lfgwcode}{label={forestexample:7}}
\documentclass[border=5pt,tikz]{standalone}
\usepackage{xltxtra}
\usepackage{forest}
\useforestlibrary{edges}
\begin{document}
\begin{forest}
 for tree={grow'=1,draw},
forked edges,
[Lebewesen
  [Menschen]
  [Tiere
    [Haustiere]
    [Raubtiere]
  ]  
  [Pflanzen]
]
\end{forest}
\end{document
\end{lfgwcode}

\begin{lfgwprint}{label={forestexample:8}}
\begin{forest}
 for tree={grow'=1,draw},
forked edges,
[Lebewesen
  [Menschen]
  [Tiere
    [Haustiere]
    [Raubtiere]
  ]  
  [Pflanzen]
]
\end{forest}
\end{lfgwprint}


\minisec{Modifizierungen}
%TODO: forestexample:10 existiert nicht mehr => Verweis nach forestexample:9 gelöscht
In die Strukturen können zusätzliche Informationen mit den Mitteln von TikZ
eingebracht werden (wie in Beispiel~\ref{forestexample:9}). Dies ist allerdings nicht so einfach,
wie die Strukturbäume zu setzen, da es Kenntnisse des komplexen PGF/TikZ verlangt.

\begin{lfgwcode}{label={forestexample:9}}
\begin{forest}
[Nomen,name=kompositum
[Adjektiv [\emph{Rot}, tier=word]]
[Nomen,name=kopf
[Nomen [\emph{kehl}, tier=word]]
[Suffix [\emph{chen}, tier=word]]
]
]
\draw[->,red,dashed,very thick] (kopf) -| node[near start,below] {+ N} (kompositum);
\node at (current bounding box.south)
[below=1ex,draw,fill=yellow, ellipse]
{\emph{Abbildung: Ein Possessivkompositum}};
\end{forest}
\end{lfgwcode}

\begin{lfgwprint}{}
\begin{forest}
[Nomen,name=kompositum
[Adjektiv [\emph{Rot}, tier=word]]
[Nomen,name=kopf
[Nomen [\emph{kehl}, tier=word]]
[Suffix [\emph{chen}, tier=word]]
]
]
\draw[->,red,dashed,very thick] (kopf) -| node[near start,below] {+ N} (kompositum);
\node at (current bounding box.south)
[below=1ex,draw,fill=yellow, ellipse]
{\emph{Abbildung: Ein Possessivkompositum}};
\end{forest}
\end{lfgwprint}

FOREST positioniert die Knoten mittels einem rekursiven Algorithmus (genauer siehe 
\cite[Kap.\,2.4]{zivanovic:forest}). Manuelle Abänderungen sind natürlich möglich. Beispielsweise
kann mit der Option 
\texttt{s sep} der Abstand zwischen den Teilsträngen, die aus einer Wurzel kommen, 
gesteuert werden; vgl. \ref{forestexample:11} und
\ref{forestexample:12}).\footnote{\cite[Beispiel\,27]{zivanovic:forest}.}


\begin{lfgwexample}{label={forestexample:11}}
\begin{forest}
for tree={s sep=(3-level)*2mm}
[CP, for tree=draw
[DP, for tree={fill=green},l*=3
[D][NP]]
[TP,for tree={fill=yellow}
[T][VP[DP][V’[V][DP]]]]
]
\end{forest}
\end{lfgwexample}

\begin{lfgwexample}{label={forestexample:12}}
\begin{forest}
for tree={s sep=(3-level)*6mm}
[CP, for tree=draw
[DP, for tree={fill=green},l*=3
[D][NP]]
[TP,for tree={fill=yellow}
[T][VP[DP][V’[V][DP]]]]
]
\end{forest}
\end{lfgwexample}

%\measurexdistance{(!11.south east)}{(!12.south west)}{+(0,-5mm)}{below}
%\path(md2)-|coordinate(md)(!221.south east);
%\measurexdistance{(!221.south east)}{(!222.south west)}{(md)}{below}
%\measurexdistance{(!21.north east)}{(!22.north west)}{+(0,2cm)}{above}
%\measurexdistance{(!1.north east)}{(!221.north west)}{+(0,-2.4cm)}{below}
Auch die Äste können unterschiedlich gespreizt werden (\texttt{fixed edge angles}),
wie es auch möglich ist, die Kantenhöhe mit \verb|for tree={l=<Wert>cm}| zu verändern 
(Beispiel~\ref{forestexample:13})
\footnote{\cite[Beispiel~59]{zivanovic:forest}}.

\begin{lfgwexample}{label={forestexample:13}}
\begin{forest}
calign=fixed edge angles,
calign primary angle=-30,calign secondary angle=60,
for tree={l=2cm}
[CP[C][TP]]
\draw[dotted] (!1) -| coordinate(p) () (!2) -| ();
\path ()--(p) node[pos=0.4,left,inner sep=1pt]{-30};
\path ()--(p) node[pos=0.1,right,inner sep=1pt]{60};
\end{forest}
\end{lfgwexample}

\subsubsection{Klammerschreibung}

Die in der Weiterentwicklung der GG im \enquote{Minimalistischen Programm}
zentrale Operation Merge (Verkettung) erzeugt mengentheoretische Objekte, die
\enquote{strenggenommen keine Strukturbäume mehr darstellen}\footnote{Günter Grewendorf: Minimalistische Syntax. A. Francke: Tübingen und Basel 2002, S.~126}. Mit Klammerschreibungen in geschweiften (s. \ref{forestexample:12}) oder eckigen Klammern (siehe Beispiel \ref{forestexample:1} oben) können diese mithilfe der Pakete \texttt{amsmath,amssymb} und diversen Umgebungen einfach in \TeX\ gesetzt werden. Sie müssen
in eine mathematische Umgebung eingefügt werden, dies sind u.\,a.:

\begin{lstlisting}
$                 $ für kürze Formalisierungen
oder
\[                \] für längere Formalisierungen
\end{lstlisting}

\begin{lfgwprint}{label={forestexample:14}}
1. Schritt Verkettung von \emph{trinken} und \emph{Milch}:\\
$ K= \{trinkt,\{Milch,trikt\}\}\ $ \\
2. Schritt Verkettung von K und Pauline:\\
$ \{trinkt, \{Pauline, \{trinkt, \{trinkt, \{Milch, trinkt\}\}\}\}\} $
\end{lfgwprint}

\begin{lstlisting}
1. Schritt Verkettung von \emph{trinken} und \emph{Milch}:\\
$ K= \{trinkt,\{Milch,trikt\}\}\ $ \\
2. Schritt Verkettung von K und \emph{Pauline}:\\
$ \{trinkt, \{Pauline, \{trinkt, \{trinkt, \{Milch, trinkt\}\}\}\}\} $
\end{lstlisting}

Beim Setzen im mathematischen Modus werden alle Buchstaben kursiv gestellt und Leerzeichen
entfernt. Mit den Befehlen \verb|text{  }| und \verb|\enspace| kann man das abändern (s. \ref{forestexample:15}).

\begin{lfgwexample}{label={forestexample:15}}
Mary wants to know\\
i) $ [[_{wh} in which house]John lived t] $ \\
ii) $ [[_\text{wh} \enspace \text{in} \enspace \text{which} \enspace \text{house}]\enspace \text{John} \enspace \text{lived} \enspace \text{t}] $
\end{lfgwexample}

%% amsmath


%\end{document}









\subsection{Stammbäume}
\index{Stammbaum} \index{Genealogie}

Mit \paket{forest} und den zahlreichen anderen Paketen zum Zeichnen mathematisch/linguistischer
Baumstrukturen stoßen Geisteswissenschaftler bei einem häufigen Anwendungsfall schnell an eine
Grenze:

Natürliche Stammbäume sind -- auch bei einfachsten familiären Strukturen -- stets mehrfach
verzweigt: Ein Vater und eine Mutter haben gemeinsame Kinder. 
In der historiographischen Realität müssen häufig noch wesentlich komplexere Familenstrukturen
abgebildet werden.

Für die Bedürfnisse von Historikern und Familienforschern wurde das Paket \paket{genealogytree}
von Thomas~F. Sturm entwickelt.
Es bietet die Möglichkeit, praktisch beliebig komplexe Familienstrukturen als Stammbaum darzustellen;
das Aussehen der Stammbäume kann auf vielfältige Weise den eigenen Vorstellungen angepasst werden.
Das Paket ist ausgesprochen umfangreich und -- der Komplexität der Aufgabenstellung geschuldet --
seine Benutzung ist nicht ganz trivial.
Allerdings enthält es eine knapp 300-seitige, vorzügliche Dokumentation mit zahlreichen Beispielen,
mit deren Hilfe man reale Familienstrukturen  angehen kann:

Zunächst muss das Paket in der Präambel eingebunden werden:
\lstinline/\usepackage[all]{genealogytree}/.

Typische westliche Kleinfamilie:

\begin{lfgwexample}{}
    \begin{tikzpicture}
    \genealogytree{
        parent{
            p{Vater}
            p{Mutter}
            g{Kind 1}
            c{Kind 2}
        }
    }
    \end{tikzpicture}
\end{lfgwexample}



der Merowinger

\begin{lfgwcode}{}
\begin{tikzpicture}
 \genealogytree[template=signpost]{
  parent{
    c{Charibert}
    c{Guntram}
    c{Sigibert}
    g{Chilperich}
    c{Gundowald}
    parent{
      c{Theuderich}
      c{Chlodomer}
      g{Childebert}
      c{Chlothar}
      parent{
        g{Chlodwig}
        parent{
          g{Childerich}
        }
      }
    } 
  }
}
\end{tikzpicture}
\end{lfgwcode}


\begin{tikzpicture}
\genealogytree[template=signpost]{
    parent{
        c{Charibert}
        c{Guntram}
        c{Sigibert}
        g{Chilperich}
        c{Gundowald}
        parent{
            c{Theuderich}
            c{Chlodomer}
            g{Childebert}
            c{Chlothar}
            parent{
                g{Chlodwig}
                parent{
                    g{Childerich}
                }
            }
        } 
    }
}
\end{tikzpicture}


\subsection{Verfassungsschemata}


\subsection{Mindmaps}

Das Paket \paket{mindmap} ist ein Ergänzungspaket zu \paket{tikZ}, d.\,h. in der Dokumentpräambel
muss zuerst \lstinline/\usepackage{tikz}/ und danach 
\lstinline/\usetikzlibrary{mindmap}/ angegeben werden.

Mit \paket{mindmap} ist es sehr einfach möglich, Mindmaps -- bis zu einer gewissen Komplexität -- zu
erstellen und ihr Aussehen -- in gewissen Grenzen -- zu modifizieren:

\begin{lfgwcode}{}
\begin{tikzpicture}
\path[mindmap, concept color=yellow]

node[concept] {Sieben Freie Künste} [clockwise from=125] 
child { node[concept] {Trivium} [clockwise from=180]
    child { node[concept] {Grammatik} }
    child { node[concept] {Rhetorik} }
    child { node[concept] {Logik} }
} 
child { node[concept] {Quadrivium} 
    child { node[concept] {Arithmetik} }
    child { node[concept] {Geometrie} }
    child { node[concept] {Astronomie} }
    child { node[concept] {Musik} }
};

\end{tikzpicture}
\end{lfgwcode}


\begin{tikzpicture}
\path[mindmap, concept color=yellow]

node[concept] {Sieben Freie Künste} [clockwise from=125] 
child { node[concept] {Trivium} [clockwise from=180]
    child { node[concept] {Grammatik} }
    child { node[concept] {Rhetorik} }
    child { node[concept] {Logik} }
} 
child { node[concept] {Quadrivium} 
    child { node[concept] {Arithmetik} }
    child { node[concept] {Geometrie} }
    child { node[concept] {Astronomie} }
    child { node[concept] {Musik} }
};

\end{tikzpicture}


\subsection{Zeitschienen}
\index{Zeitschiene}

Das kleine Paket \paket{chronology} von Levi Wiseman eignet sich hervorragend, um ohne großen
Aufwand eine Zeitschiene zu erstellen.

\begin{lfgwexample}{}
 \begin{chronology}[5]{1930}{1950}{6cm}
     \event{1933}{Machtergreifung Hitlers}
     \event[1939]{1945}{Zweiter Weltkrieg}
 \end{chronology}
\end{lfgwexample}



\subsection{Statistiken visualisieren}

Statistische Auswertungen und die Repräsentation komplexer und umfangreichen Zahlenmaterials
gehören nur bedingt zu den Bedürfnissen von Geisteswissenschaftlern.
Deshalb wird auf Pakete wie \paket{datatool} von Nicola Talbot nur hingewiesen; mit seiner Hilfe
ist es möglich, umfangreiche Datensammlungen auf \LaTeX -Ebene zu verarbeiten und zu analysieren.
Ein typischer Anwendungsfall wäre etwa die Darstellung großer Mengen von physikalischen Messwerten etc.

Geisteswissenschaftler kommen in ihrem Arbeitsalltag wohl in drei Kontexten in die Lage,
Zahlenmaterial in Diagrammform darstellen zu wollen:

\begin{itemize}
    \item Wenn es darum geht, die Aufteilung eines gegebenen Ganzen zu einem bestimmten Zeitpunkt
        auf verschiedene Teile aufzuzeigen, eignen sich besonders \emph{Tortendiagramme}.
        Typisches Anwendungsbeispiel wäre die Verteilung von Parlamentssitzen auf die Fraktionen.
    \item \emph{Säulen-} bzw \emph{Balkendiagramme} eignen sich eher, wenn es darum geht, die
        Größenunterschiede einer -- prinzipiell nach oben offenen -- Messgröße sichtbar zu machen.
        Typischer Anwendungsfall wäre etwa die Größe des Nuklearwaffenbestandes der einzelnen
        Atommächte zu einem bestimmten Zeitpunkt. Die einzelnen Säulen werden miteinander verglichen;
        sie bilden zusammen aber kein Ganzes; die Verkleinerung des Einen bedingt keine Vergrößerung
        des Anderen. 
    \item Wenn die Entwicklung einer Messgröße im Zeitverlauf dargestellt werden soll, empfehlen
        sich in erster Linie \emph{Liniendiagramme}.
        Typisches Beispiel wäre die Entwicklung der Wahlergebnisse einer politischen Partei. 
\end{itemize}

\minisec{Tortendiagramme}
\index{Tortendiagramm}

\paket{pgf-pie} ist nur in Miktex enthalten; ansonsten problemlos nachinstallierbar von 
www.ctan.org (und Abspeichern im Arbeitsverzeichnis).

Die Verteilung der Abgeordnetenmandate in der Nationalversammlung 1919:

\begin{lfgwexample}{}
    \begin{tikzpicture}
       \pie[text=inside, sum=auto, after number=,]
           {44/DNVP, 
            19/DVP, 
            91/Zentrum,
            75/DDP,
            163/SPD,
            22/USPD,
            6/Sonstige}
    \end{tikzpicture}
\end{lfgwexample}


\minisec{Säulen- bzw. Balkendiagramme}
\index{Säulendiagramm} \index{Balkendiagramm}

Während für die Erzeugung von Tortendiagrammen ein sehr kleines und handliches Paket existiert,
müssen wir für Säulen- und Liniendiagramme auf ein wesentlich umfangreicheres System zurückgreifen:
\paket{pgfplots} von Christian Feuersänger bietet einen riesigen Funktionsumfang für die 
grafische Aufbereitung von Zahlen und mathematischen Funktionen. Seine hervorragende Dokumentation
umfasst  561 Seiten -- leider nur auf Englisch.

\paket{pgfplots} erlaubt nicht nur die Erzeugung von sehr komplexen Diagrammen, sondern auch die
Beeinflussung des Aussehens bis ins letzte Detail.

Wer öfter oder intensiver mit der Darstellung von statistischem Material zu tun hat, dem kann
man nur empfehlen, sich intensiv mit diesem Paket -- ebenso wie evtl. den PSTricks-Paketen --
zu beschäftigen.

Die folgenden Beispiele haben auch die Funktion, zu zeigen, dass trotz des riesigen Umfangs des 
Paketes mit sehr wenig Aufwand typische Anforderungen etwa im Kontext von Geschichte und Politik
erfüllt werden können.

Als erster Schritt ist in der Dokumentpräambel das Paket zu laden:
\lstinline/\usepackage{pgfplots}/.

Betrachten wir zuerst eine Übersicht über große mittelalterliche Büchersammlungen:

\begin{lfgwcode}{}
        \begin{tikzpicture}
        \begin{axis}[title=Die größten Deutschen Bibliotheken,
        ybar,
        width=12cm,
        height=5cm,
        symbolic x coords={Deutsche Nationalbibliothek,
            Staatsbibliothek zu Berlin,
            Bayerische Staatsbibliothek,
            Universitätsbibliothek Frankfurt am Main,
            Technische Informationsbibliothek Hannover,
            Universitätsbibliothek Göttingen},
        xtick=data,
        x tick label style={rotate=45, anchor=east}
        ylabel={in Mio.},
        ]
        \addplot coordinates{(Deutsche Nationalbibliothek,            30.9)
            (Staatsbibliothek zu Berlin,             23.1)
            (Bayerische Staatsbibliothek,            10.4)
            (Universitätsbibliothek Frankfurt am Main,9.4)
            (Technische Informationsbibliothek Hannover,9)
            (Universitätsbibliothek Göttingen, 8)};
        \end{axis}
        \end{tikzpicture}
\end{lfgwcode}

\begin{tikzpicture}
    \begin{axis}[title=Die größten Deutschen Bibliotheken,
    ybar,
    width=12cm,
    height=5cm,
    symbolic x coords={Deutsche Nationalbibliothek,
        Staatsbibliothek zu Berlin,
        Bayerische Staatsbibliothek,
        Universitätsbibliothek Frankfurt am Main,
        Technische Informationsbibliothek Hannover,
        Universitätsbibliothek Göttingen},
    xtick=data,
    x tick label style={rotate=45, anchor=east},
    ylabel={in Mio.},
    ]
    \addplot coordinates{(Deutsche Nationalbibliothek,            30.9)
        (Staatsbibliothek zu Berlin,             23.1)
        (Bayerische Staatsbibliothek,            10.4)
        (Universitätsbibliothek Frankfurt am Main,9.4)
        (Technische Informationsbibliothek Hannover,9)
        (Universitätsbibliothek Göttingen, 8)};
    \end{axis}
    \end{tikzpicture}


Säulendiagramme lassen sich immer dann gut verwenden, wenn Größen verglichen werden können;
das gilt z.\,B. auch für Regierungszeiten:

\begin{lfgwcode}{}
\begin{tikzpicture}
 \begin{axis}[title=Die ersten Reichsregierungen der Weimarer Republik,
              xbar,
              width=10cm,
              height=5cm,
              enlarge y limits=0.5,
              xlabel=Regierungsdauer in Tagen,
              symbolic y coords={Philipp Scheidemann,
                                 Gustav Bauer,
                                 Hermann Müller,
                                 Konstantin Fehrenbach},
              ytick=data,
              ]
        \addplot coordinates{(130,Philipp Scheidemann)
                             (277,Gustav Bauer)
                             (72,Hermann Müller)
                             (317,Konstantin Fehrenbach)};
 \end{axis}
\end{tikzpicture}
\end{lfgwcode}

\begin{tikzpicture}
    \begin{axis}[title=Die ersten Reichsregierungen der Weimarer Republik,
    xbar,
    width=10cm,
    height=5cm,
    enlarge y limits=0.5,
    xlabel=Regierungsdauer in Tagen,
    symbolic y coords={Philipp Scheidemann,
        Gustav Bauer,
        Hermann Müller,
        Konstantin Fehrenbach},
    ytick=data,
    ]
    \addplot coordinates{(130,Philipp Scheidemann)
        (277,Gustav Bauer)
        (72,Hermann Müller)
        (317,Konstantin Fehrenbach)};
    \end{axis}
\end{tikzpicture}
    

\minisec{Liniendiagramme}
\index{Liniendiagramm}

Liniendiagramme eignen sich wesentlich besser, um den zeitlichen Verlauf der Entwicklung einer
bestimmten Größe zu repräsentieren:
z.\,B. die Entwicklung der Wahlergebnisse der NSDAP während der Zeit der Weimarer Republik:

\begin{lfgwcode}{}
\documentclass{article}
\usepackage{pgfplots}
\usetikzlibrary{pgfplots.dateplot}    % zur Interpretation der Kalenderdaten

\begin{document}

\pgfplotsset{width=12cm,
             height=5cm}

\begin{tikzpicture}
  \begin{axis}[title=Die Wahlergebnisse der NSDAP während der Weimarer Republik,
               date coordinates in=x, % die x-Achse enthält die Kalenderdaten!
               xtick=data,            % nur angegebene Daten nehmen,
               xticklabel style={rotate=90},
               xticklabel={\day. \month. \year},  % Daten ordentlich formatieren...
              ]
  \addplot[mark=x]coordinates{(1924-05-04,6.5)    % Jahr-Monat-Tag, danach das Wahlergebnis
                              (1924-12-07,3)
                              (1928-05-20,2.6)
                              (1930-09-14,18.3)
                              (1932-07-31,37.3) 
                              (1932-11-06,33.1)};
  \end{axis}
\end{tikzpicture}

\end{document}
\end{lfgwcode}

Das ergibt kompiliert:

\pgfplotsset{width=12cm,
             height=5cm}

\begin{tikzpicture}
\begin{axis}[
title=Die Wahlergebnisse der NSDAP während der Weimarer Republik,
date coordinates in=x, 
xtick=data,   % nur angegebene Daten nehmen
xticklabel style={rotate=90},
xticklabel={\day. \month. \year},
]
\addplot[mark=x]coordinates{(1924-05-04,6.5)
    (1924-12-07,3)
    (1928-05-20,2.6)
    (1930-09-14,18.3)
    (1932-07-31,37.3) 
    (1932-11-06,33.1)};
\end{axis}
\end{tikzpicture}

Im historischen Bereich ist das Zusatzpaket \paket{pgfplots.dateplot} von besonderem Interesse,
das in der Präambel -- nach dem Aufruf von \lstinline/\usepackage{pgfplots}/ -- anzugeben ist: \lstinline/\usetikzlibrary{pgfplots.dateplot}/.

Dank dieser Ergänzung kann \paket{pgfplots} Kalenderdaten im Format JJJJ-MM-TT verarbeiten. 
